\chapter{Parsing Theory}\label{sec:apendix-parser}

In this chapter, we present concepts of the parsing theory in more detail. A
deeper understanding of this subject is not needed to understand our work nor
results, however, it is an important subject concerning compilers.

\begin{section}{Languages}
A large part of a compiler is dedicated to parsing text to determine if
it belongs to a certain language. A \textit{language} is nothing more than
a set of words, infinite or not, over some alphabet $\Si$. We denote that
$\sigma$ is a character of the alphabet $\Si$ by expressing it as
$\sigma \in \Si$. We can also concatenate two characters $\sigma_1 \sigma_2$
to form a word, and we say that $\sigma_1 \sigma_2 \in \Sigma^2 = \Sigma\Sigma$.
In fact, we denote the product of two languages $A$ and $B$ as being
$$AB = \conj{xy}{x \in A \text{ and } y \in B}$$
which should not be confused with the cartesian product
$$A \times B = \conj{(x, y)}{x \in A \text{ and } y \in B}$$

The Kleene Star of a certain language $A$ is defined as:
$$A^* = \bigcup_{n \geq 0} A^n$$
where $A^0 = \{\lambda\}$, the set containing the \textit{empty string} $\lambda$.
Therefore, the set of all possible words over an alphabet $\Si$ is the set
$\Sis$, and every language $L$ is a subset or equal of $\Sis$, which we will
denote as $L \ssq \Sis$. We also define $A^+ = AA^*$. Union, intersection,
and complement work the same way as in set theory. The \textit{power set}
of some set $A$ is denoted by $2^A$.

The \textit{length} of a word $x$, denoted as $|x|$, is the number of characters
in the string. If $x = \lambda$, then $|x| = 0$. It should not be confused with
the size of a set $|A|$, which is the number of elements of that set.
For more details, we suggest reading the following references: \cite{sipser2012},
\cite{lewis1998elements}, and \cite{hopcroft2001introduction}.
\end{section}

\begin{section}{Finite State Automata}

One of the simplest machines designed to parse text is the Finite State
Automata. These machines are often used to build controllers for various
applications, such as elevator state controllers, microwave control panels,
signal detectors, and so on. This is because finite state machines can be
cheaply implemented using flip-flops, an electronic device that can hold a bit
as data storage.

Finite Automata are also used in Compilers as a Lexical Analyser for
recognizing tokens.  They are very useful tools since we can express these
machines easily with \textit{regular expressions}, and then algorithms can easily
convert them into a Finite Automata and minimize the number of states in it.

\begin{definition}
A Deterministic Finite Automata (\ttt{DFA}) is defined as a 5-tuple
$(Q, \Sigma, \delta, s, F)$, where
\begin{itemize}

\item $Q$ is a finite set of states.
\item $\Sigma$ is a finite set, the language alphabet.
\item $\delta:Q\times\Sigma \longrightarrow Q$ is the transition function.
\item $s \in Q$ is the initial state of the machine.
\item $F \subseteq Q$ is the set of accept states, where the machine will
decide the input as correct if computation terminates in one of these
states.
\end{itemize}
\end{definition}

We say that given an machine $\mathcal{A}$, the \textit{language}
that it generates is denoted as $L(\mathcal{A})$. The machine accepts an input
$w \in \Sis$ if $w \in L(\mathcal{A})$. We also say that a machine
recognizes a certain language $L \ssq \Sis$ if and only if
$L(\mathcal{A}) = L$.

For an example of automation, let us assume that we want to detect whether a
certain input is a non-negative integer number. Therefore, let $\Sigma$ be the
characters in the ASCII table, let $D = \{\texttt{0}, \texttt{1}, \cdots,
\texttt{9}\}$ be the set of numerals, and $\sigma \in \Sigma$ be the current
character being read. A \textit{language} describing this subset of words
is

$$ L_1 = \conj{w \in \Sis}{w \text{ is a non-negative integer}}$$

A \texttt{flex} regular expression for it would be
$$\land\texttt{[0-9]+\$}$$
and the following automata to recognize it would be
$\mathcal{A} = (\{q_1, q_2, q_3\}, \Sigma, \delta, q_1, \{q_2\})$, where
the $\delta$ function will be defined as:
	\[
		\delta(q, \sigma) = \begin{cases}
		  q_2,& \text{if } \sigma \in D \text{ and } q \in \{q_1, q_2\} \\
		  q_3,& \text{if } \sigma \in \Sigma \text{ and } q = q_3 \\
		  q_3,& \text{if } \sigma \in \Sigma - D \text{ and } q \in \{q_1, q_2\} \\
		\end{cases} \nonumber
	\]
Figure \ref{fig:a_automata} illustrates this machine. Here we denote that
$q_i \overset{\sigma}{\longrightarrow} q_j$ when the machine is on state
$q_i$, after it reads an input $\sigma$, it will jump to the state $q_j$.

Any $\DFA$ can be easily implemented using a transition
matrix. In one dimension, we have every state in the machine, in another,
every possible character of $\Sigma$.  Therefore, by just remembering the
last state of the machine, we can look up this matrix and select what state
should the machine jump to when reading a character $\sigma$. This
yields in a $O(n)$ algorithm to parse a word $w \in \Sigma^*$ with $|w| = n$.


\begin{figure}
\begin{tikzpicture}[node distance = 3cm, auto]
  \node[state, initial] (q1) {$q_1$};
  \node[state, accepting, right of=q1] (q2) {$q_2$};
  \node[state, right of=q2] (q3) {$q_3$};

  \draw[->]   (q1) edge[above, bend left] node{$\sigma \in D$} (q2);
  \draw[->]   (q1) edge[below, bend right] node{$\sigma \in \Sigma - D$} (q3);
  \draw[->]   (q2) edge[loop above] node{$\sigma \in D$} (q2);
  \draw[->]   (q2) edge[above, bend left] node{$\sigma \in \Sigma - D$} (q3);
  \draw[->]   (q3) edge[loop above] node{$\sigma \in \Sigma$} (q3);
\end{tikzpicture}

\caption{Finite Automata recognizing $\land\texttt{[0-9]+\$}$}
\label{fig:a_automata}
\end{figure}

There is also a slight variant of this machine called Non-deterministic Finite Automata (\ttt{NFA}).

\begin{definition}
A Nondeterministic Finite Automation (\ttt{NFA}) is a 5-uple
$(Q, \Sigma, \delta, s, F)$, where
\begin{itemize}

\item $Q$ is a finite set of states.
\item $\Sigma$ is a finite set, the language alphabet.
\item $\delta:Q\times\Sigma \longrightarrow 2^Q$ is the transition function.
\item $s \in Q$ is the initial state of the machine.
\item $F \subseteq Q$ is the set of accept states, where the machine will
decide the input as correct if computation terminates in one of these
states.
\end{itemize}
\end{definition}

$\NFA$ has a feature called \textit{nondeterminism}, which means that given a
$\sigma \in \Si$ and three states $p, q, r \in Q$, it is possible two have to
or more transitions from the same state given the same character $\sigma$,
which means that $p \overset{\sigma}{\longrightarrow} q$ and $p
\overset{\sigma}{\longrightarrow} r$ can both be present in the machine. Which
route the $\NFA$ will proceed with its computation in this situation does not
matter as long as it reaches some final state $f \in F$ reading the entire
input. There are also transitions like $p \overset{\lambda}{\longrightarrow}
q$, which means that the $\NFA$ will jump from $p$ to $q$ without reading any
input. These are called $\lambda$-\textit{transitions}. It is also worth
mentioning that if a $\NFA$ is in state $q$ with next input character $\sigma$
and there are no transitions of the form $q \overset{\sigma}{\longrightarrow}
r$ or $q \overset{\lambda}{\longrightarrow} r$, then this means that the path
which it has taken from $s$ to $q$ can not lead to a final state and therefore
it must take another path. If there are no remaining paths, then the $\NFA$
should reject the input.

Interestingly, if you can present a Nondeterministic Finite Automation to a
certain language, you can also present a deterministic one for the same
language and vice versa. There are even algorithms to convert an \ttt{NFA} to
\ttt{DFA}, which we will only present in this work as the following
theorem:

\begin{theorem}\label{nfa_to_dfa}
Every \ttt{NFA} has an equivalent \ttt{DFA}
\end{theorem}
\begin{proof}
	Theorem 1.39 of \citep{sipser2012} shows the Powerset Construction, an algorithm
	that converts any \ttt{NFA} into a \ttt{DFA}.
\end{proof}

\texttt{NFA}s are interesting machines in both practical and theoretical aspects,
once they are often used to implement Regular Expressions via the Thompson
Construction \citep{dragonbook} and prove theorems related to computer-theory.
However, the implementation should be done with care, as these machines require
\textit{backtracking} because of its nondeterministic behavior, which may lead
to some unfortunate bugs \citep{PR86164}. However, both \DFA and \NFA, are not
very expressive machines. For example, consider the following language:

$$L_2 = \conj{\ttt{(}^n \ttt{)}^n}{n \geq 0}$$

This is a stripped-down version of the Parenthesis Matching Problem.
Unfortunately, it is impossible to build a finite automata
that recognizes $L_2$.

\begin{lemma}
There is no $\DFA$ recognizing $L_2$.
\end{lemma}
\begin{proof}
Let $\mathcal{A} = (Q, \Sigma, \delta, s, F)$ be such $\DFA$,
with $\Si = \{\ttt{(}, \ttt{)}\}$.
Since it is a finite automation, then $Q$ is finite, so let
$n = |Q|$. Now take the following word $w = \texttt{(}^{2n}\texttt{)}^{2n}$.
Clearly, $w \in L_2$, and therefore the automation $\mathcal{A}$ should
recognize it. Therefore, there is a sequence of states
$(p_0, p_1, \cdots, p_{4n})$ when processing $w$ from $p_0 = s$ which leads to a
final state $p_{4n}$. Furthermore, the subsequence $(p_0, p_1, \cdots, p_{2n})$ is
related to the processing of the first part of the word containing only
$\texttt{(}^{2n}$. But since $2n > n$, this means that some of these
states are repeated, and therefore there is some state $q \in Q$ and integers
$i$ and $j$ such as $0 \leq i < j \leq 2n$ where $p_i = p_j = q$,
\textit{i.e.} a circuit in the graph, when processing this first part of the $w$.
Therefore, we can split $w$ in 
$$w = xyz$$
Where $x \in \Sigma^*$ is the part of $w$ before the circuit, $y \in \Sis$ is
the part of $w$ in this circuit, and $z \in \Sis$ is the part of $w$ after the
circuit. Note that $x$ and $y$ consists of a sequence of `$\texttt{(}$'
characters, because we are processing the first $2n$ characters of $w$,
and $|y| \geq 1$ because in this circuit, some characters were
read.
Since $y$ is the part of $w$ in this circuit, then
$ xyyz $
is also recognized by $\mathcal{A}$. If $k = |y|$, this means that
the input string
$$ \texttt{(}^{n+k}\texttt{)}^n \not\in L_2$$
is also recognized by $\mathcal{A}$, implying that such machine can
not exist, and therefore there is no $\DFA$ which can detect
$L_2$.
\end{proof}

In fact, this proof is heavily based on the proof of the \textit{pumping lemma},
a powerful tool to find if a certain language is not recognizable by a
Finite Automata. Entering in details about this is beyond the scope of
this work.

\end{section}

\begin{section}{Finite Pushdown Automata}
	An extension of the \NFA is the Pushdown Finite Automata (\ttt{PDA}), which
	is, in a simplistic form, a $\NFA$ with an infinite stack. This is a very
	interesting machine to parse grammars, which we will present in Section
	\ref{cfg}.

\begin{definition}
A Nondeterministic Finite Automation (\ttt{PDA}) can be defined as a 6-uple
$(Q, \Si, \Ga, \delta, q_0, F)$, where
\begin{itemize}

\item $Q$ is a finite set of states.
\item $\Sigma$ is a finite set, the language alphabet.
\item $\Ga$ is the stack alphabet,
\item $\delta:Q\times\Sigma_\lambda \times \Ga_\lambda \longrightarrow 2^{Q \times \Ga_\lambda}$ is the transition function.
\item $q_0 \in Q$ is the initial state of the machine.
\item $F \subseteq Q$ is the set of accept states, where the machine will
decide the input as correct if computation terminates in one of these
states.
\end{itemize}
\end{definition}
Here, we denote $\Gamma_\lambda$ and $\Sigma_\lambda$ as being, respectively,
$\Gamma_\lambda = \Gamma \cup \{\lambda\}$ and $\Sigma_\lambda = \Sigma \cup
\{\lambda\}$, where $\lambda$ is the empty string. Those machines starts with
an empty stack, and it will then choose to read an input based on the following
notation given a transition from $q_i \in Q$ to $q_j \in Q$: $$\sigma, \gamma_1
\rightarrow \gamma_2$$ which is the same as $$(q_j, \gamma_2) \in \delta(q_i,
\sigma, \gamma_1)$$ meaning that on input character $\sigma \in \Si_\lambda$
and with $\gamma_1 \in \Gamma_\lambda$ on the top of the stack, it advances one
character on the input string, pop $\gamma_1$ from the stack, and push
$\gamma_2 \in \Gamma_\lambda$ on top of the stack. There are some special
cases: when $\sigma = \lambda$, then the machine will not advance on the input
string. When $\gamma_1 = \lambda$, it will not check nor pop anything from the
stack. Finally, when $\gamma_2 = \lambda$, then nothing is pushed on the stack.
Note that these three cases can happen together. This notation is enough to
describe the $\delta$ function.  For an input $w \in \Sis$ to be accepted by
such automation, there must exist at least one path from the starting state
$q_0$ to a final state $q_f \in F$ consuming the entire input string $w$, like
in a $\NFA$.

These machines are more powerful than both \ttt{DFA} and \ttt{NFA}.
For every \ttt{NFA}, there is a \ttt{PDA} that detects the same language.
Furthermore, there is a $\PDA$ detecting $L_2$, in
which we presented a proof that there is no \ttt{DFA} for it.

\begin{figure}
\begin{tikzpicture}[node distance = 3cm, auto]
  \node[state, initial] (q1) {$q_1$};
  \node[state, right of=q1] (q2) {$q_2$};
  \node[state, right of=q2] (q3) {$q_3$};
  \node[state, accepting, right of=q3] (q4) {$q_4$};

  \draw[->]   (q1) edge[above, bend left] node{$\lambda, \lambda \rightarrow \$$} (q2);
  \draw[->]   (q2) edge[above, bend left] node{$\lambda, \lambda \rightarrow \lambda$} (q3);
  \draw[->]   (q2) edge[loop below] node{$\ttt{(}, \lambda, \rightarrow \ttt{(}$} (q2);
  \draw[->]   (q3) edge[above, bend left] node{$\lambda, \$ \rightarrow \lambda$} (q4);

  \draw[->]   (q3) edge[loop below] node{$\ttt{)}, \ttt{(} \rightarrow \lambda$} (q3);
\end{tikzpicture}

\caption{Finite Automata recognizing $L_2$}
\label{fig:pda_l2}
\end{figure}

Figure \ref{fig:pda_l2} describes a $\PDA$ which detects $L_2$.
A sketch of why this machine works is as follows: on the transition
from $q_1$, to $q_2$, the machine will push an token $\$ \in \Ga$ to mark the
beginning of the stack. Then we may proceed reading $\ttt{(}$ through the loop
at $q_2$, and pushing then on the stack, or advance without reading, poping or
pushing anything on the stack.  Once all `$\ttt{(}$'s have been pushed on the
stack, it will then proceed reading all `$\ttt{)}$' while poping the `$\ttt{(}$'
characters of the stack to count how many occurrences it has seen so far.
If the stack ends empty after this process, it means that the number of `$\ttt{(}$'
is equal to the number of `$\ttt{)}$'. If the machine incorrectly guesses when to switch
from $q_2$ to $q_3$ even if $w \in L_2$, this would generate an invalid path
because the top of the stack would contain a `$\ttt{(}$' and not the $\$$, and
the same goes when $w = \ttt{(}^i\ttt{)}^j$, with $j < i$. If, however, $i >
j$, then it is possible to jump from $q_3$ to $q_4$, however, the entire input string
would not be completely consumed, and therefore the machine will reject the
string. The only way that an string is accepted is when $j = i$.

Now that we showed one language that a \ttt{PDA} can recognize that a
\ttt{NFA} can not, we may prove that any language recognizable by an
\ttt{NFA} is recognizable by a \ttt{PDA}.

\begin{theorem}
Let $L$ be a language recognizable by an \ttt{NFA}. Then there is a \ttt{PDA}
which detects $L$.
\end{theorem}
\begin{proof}
	Let $\mathcal{A} = (Q, \Si, \delta, s, F)$ be a \ttt{NFA} recognizing $L$.
We will show how to build a \ttt{PDA} $\mathcal{P}$ that recognizes $L$.
The idea is that a $\ttt{PDA}$ that does not use its stack is just a \ttt{NFA}.

Let $\mathcal{P} = (Q, \Si, \emptyset, \delta_p, s, F)$, with $Q, \Si, s, F$ being
exactly the same as in $\mathcal{A}$. Now let $\delta_p$ be such that
$$(q_j, \lambda) \in \delta_p(q_i, \sigma, \lambda) \text{ if and only if } q_j \in \delta(q_i, \sigma)$$
where $\sigma \in \Sigma_\lambda$ and $q_i, q_j \in Q$. This means that given an
input string $w \in L(\mathcal{A})$, its sequence of states $(p_0, p_1, \cdots, p_{|w|})$
with $p_0 = s$ and $p_{|w|} \in F$, is also a valid sequence of states in $\mathcal{P}$,
implying that $w \in L(\mathcal{P})$, and vice versa. This means that $\mathcal{A}$ and
$\mathcal{P}$ recognizes the same language.
\end{proof}

However, as these machines can use \textit{nondeterminism} to find its path
to a final state, its implementation may require \textit{backtracking},
and in fact, the $\PDA$ in Figure \ref{fig:pda_l2} will require it to correctly
``guess'' when to jump from $q_2$ to $q_3$. We can eliminate the possibility
of backtracking if we remove the nondeterminism from the machine.

\end{section}

\begin{section}{Deterministic Finite Pushdown Automata}
	We may now present a restriction of the \ttt{PDA} so that
it is possible to remove the possibility of backtracking from these machines.
The way to do it is to restrict when the machine can have certain transitions
with regard to $\delta(q, \sigma, \lambda)$, $\delta(q, \lambda, \gamma)$ and
$\delta(q, \lambda, \lambda)$.

\begin{definition}
A Deterministic Finite Automation \ttt{DPDA} can be defined as a 6-uple
$(Q, \Si, \Ga, \delta, q_0, F)$, where
\begin{itemize}

\item $Q$ is a finite set of states.
\item $\Sigma$ is a finite set, the language alphabet.
\item $\Ga$ is the stack alphabet,
\item $\delta:Q\times\Sigma_\lambda \times \Ga_\lambda \longrightarrow Q \times \Ga_\lambda \cup \{\emptyset \}$ is the transition function.
\item $q_0 \in Q$ is the initial state of the machine.
\item $F \subseteq Q$ is the set of accept states, where the machine will
decide the input as correct if computation terminates in one of these
states.
\end{itemize}

Furthermore, $\delta$ must also satisfy the following condition: For every $q \in Q$, $\sigma \in \Si$,
and $\gamma \in \Ga$, one and only one of these values
$$\delta(q, \sigma, \gamma), \delta(q, \lambda, \gamma), \delta(q, \sigma, \lambda), \delta(q, \lambda, \lambda)$$
is not $\emptyset$.

\end{definition}

By setting these restrictions on $\delta$, we prevent the machine from taking two
distinct actions given the same input, as it is possible in \ttt{PDA}
by setting both $\delta(q, \sigma, \gamma) \neq \emptyset$ and
$\delta(q, \sigma, \lambda) \neq \emptyset$. The machine can also only take a
move of form $\delta(q, \sigma, \lambda)$ or $\delta(q, \lambda, \lambda)$ when
the stack is empty, otherwise the machine will always reject the input. Acceptance
happens the same way as in \ttt{PDA}.

We may now show a $\DPDA$ for $L_2$. We can start
from the nondeterminisitc version, since the only condition that is violated
there with regard to a $\DPDA$ are the transitions
$\delta(q_2, \ttt{(}, \lambda)$ and $\delta(q_2, \lambda, \lambda)$: only one
of these transitions can exist. We therefore can design the following machine
to address these problems, eliminating the $\delta(q_2, \lambda, \lambda)$
transition and adding an extra final state so that it accepts $\lambda \in L_2$.

\begin{figure}
\begin{tikzpicture}[node distance = 3cm, auto]
  \node[state, accepting, initial] (q1) {$q_1$};
  \node[state, right of=q1] (q2) {$q_2$};
  \node[state, right of=q2] (q3) {$q_3$};
  \node[state, accepting, right of=q3] (q4) {$q_4$};

  \draw[->]   (q1) edge[above, bend left] node{$\lambda, \lambda \rightarrow \$$} (q2);
  \draw[->]   (q2) edge[above, bend left] node{$\ttt{)}, \ttt{(} \rightarrow \lambda$} (q3);
  \draw[->]   (q2) edge[loop below] node{$\ttt{(}, \lambda, \rightarrow \ttt{(}$} (q2);
  \draw[->]   (q3) edge[above, bend left] node{$\lambda, \$ \rightarrow \lambda$} (q4);

  \draw[->]   (q3) edge[loop below] node{$\ttt{)}, \ttt{(} \rightarrow \lambda$} (q3);
\end{tikzpicture}

\caption{Finite Automata recognizing $L_2$}
\label{fig:pushdown_automata}
\end{figure}

One can clearly see that such machines can be implemented with an $O(n)$ algorithm,
once backtracking is not needed. Unfortunately, such machines are less powerful
than the Nondeterministic version. For instance, the language
$$L_3 = \conj{ww^R}{w \in \{0, 1\}^*}$$
can not be detected by such machines, but one can clearly design a $\PDA$
that recognizes this language: just push a certain amount of characters
and nondeterministically ``guess'' when to start popping the characters
from the stack and comparing to the next input characters.

\begin{lemma}\label{lemma:pushdown_1}
	There is no Deterministic Finite Pushdown Automata that detects $L_3$
\end{lemma}
Proof of such lemma is quite complicated, however the idea behind it is that
there is no way that the machine find out when to stop pushing the symbols
on the stack and start popping the symbols to compare for equality, once
there is no way to discover the point exactly when $w$ stops and $w^R$ starts
because the machine lacks nondeterminism. This, however, shows that there
are Languages that can be detected with Nondeterministic Finite Pushdown
Automata.

\begin{lemma}\label{lemma:pushdown_2}
	For every \ttt{DPDA} $\mathcal{D}$, there is
	an equivalent \ttt{PDA} $\mathcal{P}$ that recognizes the
	same language.
\end{lemma}
\begin{proof}
	 It is enough to note that any \ttt{DPDA} is a \ttt{PDA}.  Let $\mathcal{D}
	 = (Q, \Si, \Ga, \delta, s, F)$ a \DPDA. Now define
	 $\delta_p : Q\times\Si_\lambda\times\Ga_\lambda \longrightarrow 2^{Q\times\Ga_\lambda}$ such that

	 $$ (q_j, \gamma_2) \in
	 \delta_p(q_i, \sigma, \gamma_1) \text{ if and only if } \delta(q_i,
	 \sigma, \gamma_1) = (q_j, \gamma_2)$$

	 Where $\sigma \in \Si_\lambda$, and
	 $\gamma_1, \gamma_2 \in \Ga_\lambda$. Then $\mathcal{P} = (Q, \Si, \Ga,
	 \delta_p, s, F)$ recognizes the same language as $\mathcal{D}$.
\end{proof}

Lemma \ref{lemma:pushdown_1} and \ref{lemma:pushdown_2} shows that \ttt{DPDA} are not
so powerful when compared to \ttt{PDA}.

\end{section}

\begin{section}{Context-Free Grammars}\label{cfg}

Grammars are commonly used to express natural language. However, there is
a subset of grammars that are of interest when designing
Programming Languages, which are called Context-Free Grammars.

Context-Free Grammars are a very powerful notation to express how a Programming Language
should behave when reading a program as input, however, algorithms that parse
such grammars, in general, are $O(n^3)$, where $n$ is the size of the input. Hopefully,
if we restrict the grammar to be a Deterministic Context-Free Grammar, we can ensure
that parsing can take $O(n)$.

\begin{definition}
A Context-Free Grammar $G = (V, \Si, R, S)$ is a 4-uple where:
\begin{itemize}
	\item $V$ is the set of \textit{Variables}, often named set of
	\textit{Nonterminal} symbols.
	\item $\Si$ is the alphabet, often named \textit{terminal} symbols.
	\item $R \ssq V \times (\Si \cup V)^*$ is the set of \textit{derivation rules}.
	\item $S \in V$ is the \textit{starting variable}
	\item $V \cap \Si = \varnothing$
\end{itemize}
\end{definition}

Furthermore, we also define

\begin{definition}
	If $A \in V$ derivate a string $x \in (V \cup \Si)^*$ in exactly one step, then
	we denote $A \Rightarrow x$. If we can derive in zero or more steps, then
	$A \Rightarrow^* x$. If in one or more steps, $A \Rightarrow^+ x$
\end{definition}

For instance of grammar, $G = (V, \Si, R, S)$ such that
$V = \{S\}$, $\Si = \{\ttt{(}, \ttt{)}\}$, and $R$ is defined such that
$$S \rightarrow \ttt{(}S\texttt{)} \; | \; \lambda$$
can recognize $L_2 = \conj{\ttt{(}^n \ttt{)}^n}{n \geq 0}$, because it is clear that
$S \Rightarrow^{n+1} \ttt{(}^n\ttt{)}^n$.

There is also an important undesired detail about grammar called ambiguity, which
allows a string to have multiple derivations. This can confuse a parser into
generating a distinct parse tree from the one that is intended by the grammar
designer.

\begin{definition}
	A grammar $G$ is ambiguous if there are multiple leftmost derivations for the same
	word.
\end{definition}
\begin{definition}
	A leftmost derivation of a word $w$ from a grammar $G$ is a derivation
	in which we always derive the leftmost rule.
\end{definition}

For instance, let $G = (\{S\}, \{a, +, \times\}, R, S)$, where $R$ is defined
as
$$S \rightarrow S+S \; | \; S \times S \; | \; a $$
it is clear that $S \Rightarrow^* w$ if $w$ is a arithmetic expression. This
grammar is ambiguous, as there are two leftmost derivations for the same string.
One derivation is
$$ S \Rightarrow S + S \Rightarrow a + S \Rightarrow a + S \times S \Rightarrow a + a \times S \Rightarrow a + a \times a$$
Another is
$$ S \Rightarrow S \times S \Rightarrow S + S \times S \Rightarrow a + S \times S \Rightarrow a + a \times S \Rightarrow a + a \times a$$

This may result in a troublesome parsing, for instance, consider the
following sketch of Grammar:

% Avoid bug in vim 8.2
%This is here because VIM coloring glitches with this input.

\begin{align}
\textit{stmt} &\rightarrow \textbf{if } \textbf{(}\textit{expr}\textbf{) } \textit{stmt} \\
\textit{stmt} &\rightarrow \textbf{if } \textbf{(}\textit{expr}\textbf{) } stmt \textbf{ else } \textit{stmt}
\end{align}


This can generate the two C code snippets in Figure \ref{fig:left_if} and
\ref{fig:right_if}. In C, spaces and tabulations are ignored if they are
repeated, and therefore these two codes are the same. Therefore, that
\textbf{else} could belong to both first and last \textbf{if} because of
ambiguity. 

\begin{figure}[ht]
    \centering
    \begin{subfigure}[b]{0.40\textwidth}

		\begin{lstlisting}[
			language=pseudocode,
			style=pseudocode,
			style=wider,
			functions={},
			specialidentifiers={},
			]
			if (a)
				if (b)
					statement;
				else
					another_statement;
		\end{lstlisting}
        \caption{\label{fig:left_if}}

    \end{subfigure}
    \begin{subfigure}[b]{0.40\textwidth}
		\begin{lstlisting}[
			language=pseudocode,
			style=pseudocode,
			style=wider,
			functions={},
			specialidentifiers={},
			]
			if (a)
				if (b)
					statement;
			else
				another_statement;
		\end{lstlisting}
        \caption{\label{fig:right_if}}
\end{subfigure}
\caption{Distinct C parse tree for the same input}
\end{figure}

However, the C specification states that an else should always match with the closest
if, and therefore Figure \ref{fig:left_if} must be selected
when parsing a C program, or else the compiler has a bug. This can lead
to some unfortunate bugs in the code if the programmer is not careful enough,
because one may write the code in \ref{fig:right_if} expecting that the
tabulation will change the program semantic, and now you have a bug which is
very hard to catch \citep{ritchie1988c}. This issue can be avoided by removing the
ambiguity of the grammar. \cite{dragonbook} presents the following grammar
to fix this problem for a $\LR{1}$ parser:
\vspace*{-0.5cm}
%This is here because VIM coloring glitches with this input.

\begin{align}
\textit{stmt} &\rightarrow \textit{matched\_stmt} \text{ | } \textit{open\_stmt} \\
\textit{matched\_stmt} &\rightarrow \textbf{if } \textbf{(}\textit{expr}\textbf{) } \textit{matched\_stmt} \textbf{ else } \textit{matched\_stmt} \\
\textit{open\_stmt} &\rightarrow \textbf{if } \textbf{(}\textit{expr}\textbf{) } \textit{stmt} \\
&\phantom{\rightarrow}| \hspace{0.15cm} \textbf{if } \textbf{(}\textit{expr}\textbf{) } \textit{matched\_stmt} \textbf{ else } \textit{open\_stmt}
\end{align}

however, one can design a Top-Down Recursive-Descent parser that correctly
parses that grammar, even with the ambiguity.

We may now define what it means for a Grammar to be ``Deterministic'' by
defining the \textit{handle} of a string.

\begin{definition}
	Let $G = (V, \Si, R, S)$ be a Context-Free Grammar that Recognizes $L$.
	The \textbf{handle} of a string $w$ such that $S \Rightarrow^* w$ is a string
	$h \in (V \cup \Si)^*$ such that
	\begin{enumerate}
	  \item $w = xhy$, where $x \in (V \cup \Si)^*, A \Rightarrow h, A \in V$,
	  $y \in \Sis$
	  \item $h$ is the leftmost string satisfying condition 1.
	\end{enumerate}
\end{definition}

One can clearly see that if $G$ is unambiguous, then $w$ should have a unique
handle, otherwise there would be another leftmost derivation of $w$ from $S$, which
contradicts the definition of unambiguity.

Furthermore, if we enforce that given a handle of a word $w = xhy$
is also the handle of $xhz $ for all $z \in \Sis$ such that $S \Rightarrow^* w$ and
$S \Rightarrow^* xhz$, then we have an entire
class of languages called \textit{Determinisitic Context-Free Languages}

\begin{definition}
A Deterministic Context-Free Grammar (\ttt{DCFG}) $G = (V, \Si, R, S)$ is a 4-uple where:
	\begin{itemize}
		\item $(V, \Si, R, S)$ is a $\CFG$.
		\item Given $w = xhy$ such that $S \Rightarrow^* w$, where $h$ is the handle
		of $w$, then $h$ is the handle for all words $xhz \in L$, where $z \in \Sis$.
	\end{itemize}
\end{definition}

$\DCFG$ are also called \ttt{LR}(0) Grammars, because it can
be easily parsed from \texttt{L}eft-to-right, always taking the \texttt{R}ightmost
derivations (\textit{i.e.} leftmost reductions).

\end{section}
\begin{section}{Bottom-up Parsing}\label{sec:bottom-up}

With the idea of handles in mind, it is possible to design a greedy algorithm
for reducing an input string $w$ of a $\DCFG$ $G = (V, \Sigma, R, S)$. Let $w = xhy$, where $h$ is the
handle. Then there is some rule $\left(A \rightarrow h \right) \in R$, which will lead to the
string $xAy$. Note that the string $y$ was never read so far. This new string
may have another handle, say $h_2$ in $xA$, in which we \textbf{reduce} again
using some production rule, or we may need to \textbf{shift} to the next character
in trying to find the handle for the string. Therefore, we may get into a state
$$u \bullet v$$
where $u \in (V \cup \Si)^*$ are the symbols currently \textit{on the stack},
and $v \in \Sis$ are the characters in the input, which we never read so far.
Note that $u$ may contain Nonterminal characters, because we may have already
reduced strings before the $\bullet$ symbol.

Once reading a character, we are interesting in how to detect that we have
seen a handle to reduce. Therefore, we are interested in the elements of
the following set:

$$H_0 = \conj{xh}{S \Rightarrow^+ xhy, h \text{ is the handle of } xhy, x \in (V \cup \Si)^*, y \in \Sis}$$

\cite{knuth1965translation} showed that this set is a \textit{regular language},
meaning that there is a $\DFA$, say $\mathcal{DK} = (Q, V \cup \Si, \delta, s, F)$, which
recognizes $H_0$. Notice that $\mathcal{DK}$ runs on alphabet $V \cup \Si$.
He also shows that every \textit{accept} state of the $\mathcal{DK}$ is related
to a production rule of $G$, in which the algorithm should use to reduce the
contents of the stack.

Furthermore, \cite{sipser2012} shows that the construction of the
\DFA can be used to test if the grammar is \DCFG by tracking two
conflicts in its state machine:

\begin{itemize}
	\item Shift/Reduce conflicts, when the parser can not decide when
	to Shift (i.e. push next token onto stack) or to Reduce based on
	some production rule.

	\item Reduce/Reduce conflicts, when the parser can not decide which
	rule to reduce to because there are two or more production rules
	related to one $\mathcal{DK}$ accept state.
\end{itemize}

Implementation of this algorithm consists in showing a \DPDA that
can be used to parse the input string. When reading the input string,
the \DPDA also simulates a single character read of $\mathcal{DK}$,
and push the current $\mathcal{DK}$ state on the stack. If
$\mathcal{DK}$ reaches one of its accept state, it means that we
found a handle associated with some production $A \rightarrow h$,
in which we pop $|h|$ states from the \DPDA stack and ``emulates''
the read of $A$ as the next read character, once a \DPDA can
not return a character to its input, and proceed reading the input.
If it reads the entire input string and ``emulate'' the reduction
of the starting variable, it means that it is finished. This also
shows that
\begin{lemma}
For every \DCFG there is an equivalent \DPDA
\end{lemma}

This before-mentioned \DPDA is the $\LR{0}$ algorithm.
However, this algorithm is not so powerful just because it relies on
\DCFG. For instance, this simple grammar $G_A = (V, \Si, R, S)$ such that
$V = \{S, E, T\}, \Si = \{a, \times, +\}$, and $R$ be defined as the rules
\begin{align}
S &\rightarrow E \nonumber \\
E &\rightarrow E + T \; | \; T \nonumber \\
T &\rightarrow T \times a \; | \; a \nonumber
\end{align}

\begin{lemma}\label{shift_reduce}
	$G_A$ is not $\ttt{LR}(0)$.
\end{lemma}
\begin{proof}
	Let $u = T \times a$ and $v = T$. Clearly, $S \Rightarrow^* u$ and
	$S \Rightarrow^*v$. Since $u = v \times a$, then the handle of
	$v$ must be equal the handle of $u$. However, the handle of $v$
	is $E$ and the handle of $u$ is $T$, which means that $G$ is not
	\ttt{DCFG} and therefore can not be parsed with a $\LR{0}$
	algorithm.
\end{proof}

This means that this algorithm can not be used to parse even very simple
grammars such as above, it will always face a Shift/Reduce conflict
because on state $T\bullet$ it should reduce to $E$, however, when
facing the input $T\bullet \times a$, it must shift.

Fortunately, there are ways to expand this algorithm.
Instead of only using the symbols that are currently on the stack for
determining which rule to reduce, we could use the immediate next
token on the input, the \textit{lookahead}, to decide what to do, \textit{i.e.},
if shift or reduce. So, for instance, when facing the state
$T\bullet \times a$, the parser should \textit{shift}, because
$\times$ is the current lookeahead token, even if $T \in H_0$.
However, on state $T\bullet + a$ and $T\bullet$, the parser should
\textit{reduce} with regard to $E \rightarrow T$. This is exactly the
idea behind the $\LR{1}$ parser. In fact, we can define

\begin{definition}
A $\CFG$ $G = (V, \Si, R, S)$ is $\LR{1}$ if given a string
$w$ such that $w\$ = xhy \$$ with $S \Rightarrow^* w$, $h$ as
handle, $x \in (V \cup \Si)^*$, $y \in \Sis$, and $ \$ \not\in \Si \cup V$,
then $h$ is also the handle of $xhz\$ $ if shares the same first character of $y$, i.e.,
$y = \sigma u$ and $z = \sigma v$.
\end{definition}

Therefore, when building the $\ttt{NFA}$ $\mathcal{DK}$, it must now also
consider the first character immediately next of $\bullet$ in its computation.
That is why the character $\$$ was added to the input string.
With this change, \cite{sipser2012} proves that $G_A$ can be parsed with a
$\LR{1}$ algorithm.

Finally, these tests are widely implemented in \ttt{LR}(1) parser generators
such as the GNU Bison \citep{manual21bison}. \ttt{LR}(1) parsers are widely
used in practice, together with \texttt{LALR}(1) bottom-up parser and the top-down,
Recursive-Descent parser.

\end{section}

\begin{section}{Top-Down Parsing}

Another way to parse a Context-Free Grammar is, instead of on input string $w$
try to reduce back to the starting rule $S$, try to derive $w$ from $S$.
A very common way of doing this in practice is to employ an algorithm called
Recursive-Descent.

The Recursive-Descent is a technique used to parse any generic Context-Free
Grammar that is free from left recursion, however \textit{backtracking} is
necessary in several cases. It is also very simple to understand, as there is
less computer-theory background necessary to understand how they operate and to
implement by hand, without the use of any parse-generator tools. In fact, this
is the parser algorithm used by GCC when compiling C \citep{myers_parser}, as
well as the parser used in clang \citep{clang_parser}. 

Let $G = (V, \Si, R, S)$ be a Context-Free Grammar. In a Recursive Descent
Parser, each $v \in V$ are represented as a function in the algorithm, and
that function is responsible to correctly match it derivation by looking
the next input tokens. Here we are not limited to just one token of lookahead,
however, the programmer should keep in mind that the more complex method
he uses to determine which rule to derive to, the more he adds to the
time complexity of the parser, and therefore it may not operate on $O(n)$,
where $n$ is the size of the input.

Here is an example of Recursive-Descent Parser. Let $G_A^2 = (V, \Si, R, S)$,
where $V = \{S, E, T\}, \Si = \{+, \times, a\}$ and $R$ be defined as

\begin{align}
S &\rightarrow E \nonumber \\
E &\rightarrow T + E \; | \; T  \nonumber \\
T &\rightarrow a \times T \; | \; a \nonumber
\end{align}

Note that $G_A^2$ differs from $G_A$ just by the recursions in $E$ and $T$. We
will show what happens if one tries to parse $G_A$ on these algorithms in the
sequence, but we present in Figure \ref{fig:recursive-descent} a pseudocode for
an algorithm that parses $G_A^2$. Note that the programmer must also take care
of handling the syntax errors, and therefore this task can become very
error-prone.


\begin{figure}[ht]
	\centering
	\begin{lstlisting}[
		language=pseudocode,
		style=pseudocode,
		style=wider,
		functions={},
		specialidentifiers={extern, TOKEN},
		]
		extern function syntax_error ();
		extern function yylex ();

		TOKEN current_token;
		function next_token ()
			current_token = yylex ();

		function T ()
			next_token ();

			if (current_token != 'a')
				syntax_error ();

			next_token ();
			if (current_token = '*')
				T ();

		function E ()
			T ();
			if (current_token = '+')
				E ();
				return;
			
			else if (current_token != EOF)
				syntax_error ();
		
		function S ()
			E ();
	\end{lstlisting}
\caption{Recursive-Descent parser for $G_A^2$}
\label{fig:recursive-descent}
\end{figure}

Algorithm in Figure \ref{fig:recursive-descent} is able to recognize $G_2^A$.
Now, what if we try to deploy such a parsing algorithm for $G_2^A$? Figure
\ref{fig:recursive_descent_2} presents what would be the function \texttt{E()}.
Notice that this function will call itself again right after it has been
called. This is an infinite recursion that will lead the parser to an infinite
loop or a crash due to a runtime stack overflow. Therefore, to deploy a
Recursive-Descent parser, the following definition must hold for the grammar.

\begin{figure}[ht]
	\centering
	\begin{lstlisting}[
		language=pseudocode,
		style=pseudocode,
		style=wider,
		functions={},
		specialidentifiers={extern},
		]
		function E ()
			E ();
			if (current_token == '+')
				T ();
				return;
			
			else if (current_token != EOF)
				syntax_error ();
	\end{lstlisting}
\caption{Parsing code for nonterminal $\protect E \in V$ of $\protect G_A$ }
\label{fig:recursive_descent_2}
\end{figure}

\begin{definition}
	Let $G = (V, \Si, R, S)$ be a Context-Free Language. $G$ is Free of
	Left Recursions if, for every $A \in V$, there is no derivations
	of the form
	$$ A \Rightarrow^+ A \alpha$$
	for any $\alpha \in (V \cup \Sigma)^*$
\end{definition}

Fortunately, there are algorithms for eliminating the left-recursion of
\textit{any} \texttt{CFG}, but they are often eliminated by hand redesigning
the grammar to avoid such problems, once $\ttt{LR}(1)$ parsers can also handle
right recursions very well \citep{dragonbook}.

There are also conditions in which a Recursive-Descent can be implemented
without backtracking, which are the class of $\ttt{LL}(k)$ parsers. We
won't discuss these parsers as they are not powerful as powerful
as \LR{$k$} parsers, and they are not as well used in pratice.

\end{section}
