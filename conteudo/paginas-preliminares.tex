%!TeX root=../tese.tex
%("dica" para o editor de texto: este arquivo é parte de um documento maior)
% para saber mais: https://tex.stackexchange.com/q/78101/183146

%%%%%%%%%%%%%%%%%%%% DEDICATÓRIA, RESUMO, AGRADECIMENTOS %%%%%%%%%%%%%%%%%%%%%%%

%\begin{dedicatoria}
%Esta seção é opcional e fica numa página separada; ela pode ser usada para
%uma dedicatória ou epígrafe.
%\end{dedicatoria}

% Reinicia o contador de páginas (a próxima página recebe o número "i") para
% que a página da dedicatória não seja contada.
\pagenumbering{roman}

% Agradecimentos:
% Se o candidato não quer fazer agradecimentos, deve simplesmente eliminar
% esta página. A epígrafe, obviamente, é opcional; é possível colocar
% epígrafes em todos os capítulos. O comando "\chapter*" faz esta seção
% não ser incluída no sumário.
\chapter*{Acknowledgments}
\epigrafe{If I have seen further it is by standing on the shoulders of Giants}{Isaac Newton}

This thesis would not be possible without the help of my family, and several
persons that I came across in my life. First my journey to the university, to
my mother Débora Faulin, and father Antonio Silvio Belinassi, which could
provide me good education to join one of the best universities in my home
country; then my aunt Delaine Faulin and my cousin Luccas Doi, which could
provide me a second home to attend the university. Without their love and
support, definitively this thesis would never be possible.

In the university, there were several professors which inspired me to get into
High-Performance Computing, mainly Marco Gubitoso, Alfredo Goldman, and Ernesto
Birgin. Furthermore, one of the key person into this thesis was a friend,
Rodrigo Siqueira, which helped me constructing a path which led to contributing
to GCC and meet, even if through only in virtual ways, to some of the greatest
minds in Compilers. One of them, Richard Biener, had huge contributions to this
thesis with both his ideas and codebase knowledge. For these persons, I can
only show my deep acknowledgments.

Furthermore, I thank my laboratory colleagues for their companies, help, and
suggestions, which include Rodrigo Siqueira himself, Nelson Lago, Matheus
Tavares, Dylan Guedes, Tallys Martins, Melissa Wen, Fernanda Magano, Fernando
Freire, Marcelo Schmitt, and Ederson Ferreira.

% Resumo e abstract são definidos no arquivo "metadados.tex". Este
% comando também gera automaticamente a referência para o próprio
% documento, conforme as normas sugeridas da USP.
\printResumoAbstract


%%%%%%%%%%%%%%%%%%%%%%%%%%% LISTAS DE FIGURAS ETC. %%%%%%%%%%%%%%%%%%%%%%%%%%%%%

% Como as listas que se seguem podem não incluir uma quebra de página
% obrigatória, inserimos uma quebra manualmente aqui.
\makeatletter
\if@openright\cleardoublepage\else\clearpage\fi
\makeatother

% Todas as listas são opcionais; Usando "\chapter*" elas não são incluídas
% no sumário. As listas geradas automaticamente também não são incluídas
% por conta das opções "notlot" e "notlof" que usamos mais acima.

% Normalmente, "\chapter*" faz o novo capítulo iniciar em uma nova página, e as
% listas geradas automaticamente também por padrão ficam em páginas separadas.
% Como cada uma destas listas é muito curta, não faz muito sentido fazer isso
% aqui, então usamos este comando para desabilitar essas quebras de página.
% Se você preferir, comente as linhas com esse comando e des-comente as linhas
% sem ele para criar as listas em páginas separadas. Observe que você também
% pode inserir quebras de página manualmente (com \clearpage, veja o exemplo
% mais abaixo).
\newcommand\disablenewpage[1]{{\let\clearpage\par\let\cleardoublepage\par #1}}

% Nestas listas, é melhor usar "raggedbottom" (veja basics.tex). Colocamos
% a opção correspondente e as listas dentro de um grupo para ativar
% raggedbottom apenas temporariamente.
\bgroup
\raggedbottom

%%%%% Listas criadas manualmente

%\chapter*{Lista de Abreviaturas}
\disablenewpage{\chapter*{Lista de Abreviaturas}}

\begin{tabular}{rl}
	GCC & GNU Compiler Collections \\
	LTO & Link Time Optimization \\
	URL & Localizador Uniforme de Recursos (\emph{Uniform Resource Locator})\\
	IME & Instituto de Matemática e Estatística\\
	USP & Universidade de São Paulo \\
	DFA & Deterministic Finite Automata \\
	NFA & Nondeterministic Finite Automata \\
	PDA & Nondeterministic Finite Pushdown Automata \\
	DPDA & Deterministic Finite Pushdown Automata \\
	CFG & Context-Free Grammar \\
	CFGraph & Control-Flow Graph \\
	IPA & Interprocedural Analysis \\
	AST & Abstract Syntax Tree \\
	IR  & Intermediate Representation \\
	API & Application Programming Interface \\
	EH  & Exception Handles \\

\end{tabular}

%\chapter*{Lista de Símbolos}
\disablenewpage{\chapter*{Lista de Símbolos}}

\begin{tabular}{rl}
        $\Sigma$    & Language Alphabet\\
        $\Gamma$    & Machine Alphabet\\
		$\delta$    & Transition Function\\
		$\Sigma_{\lambda}$ & Language Alphabet with the emptry string\\
		$\Gamma_{\lambda}$ & Machine Alphabet with the emptry string\\

\end{tabular}

% Quebra de página manual
\clearpage

%%%%% Listas criadas automaticamente

% Você pode escolher se quer ou não permitir a quebra de página
%\listoffigures
\disablenewpage{\listoffigures}

% Você pode escolher se quer ou não permitir a quebra de página
%\listoftables
\disablenewpage{\listoftables}

% Esta lista é criada "automaticamente" pela package float quando
% definimos o novo tipo de float "program" (em utils.tex)
% Você pode escolher se quer ou não permitir a quebra de página
%\listof{program}{\programlistname}
\disablenewpage{\listof{program}{\programlistname}}

% Sumário (obrigatório)
\tableofcontents

\egroup % Final de "raggedbottom"

% Referências indiretas ("x", veja "y") para o índice remissivo (opcionais,
% pois o índice é opcional). É comum colocar esses itens no final do documento,
% junto com o comando \printindex, mas em alguns casos isso torna necessário
% executar texindy (ou makeindex) mais de uma vez, então colocar aqui é melhor.
\index{Inglês|see{Língua estrangeira}}
\index{Figuras|see{Floats}}
\index{Tabelas|see{Floats}}
\index{Código-fonte|see{Floats}}
\index{Subcaptions|see{Subfiguras}}
\index{Sublegendas|see{Subfiguras}}
\index{Equações|see{Modo Matemático}}
\index{Fórmulas|see{Modo Matemático}}
\index{Rodapé, notas|see{Notas de rodapé}}
\index{Captions|see{Legendas}}
\index{Versão original|see{Tese/Dissertação, versões}}
\index{Versão corrigida|see{Tese/Dissertação, versões}}
\index{Palavras estrangeiras|see{Língua estrangeira}}
\index{Floats!Algoritmo|see{Floats, Ordem}}
