% O resumo é obrigatório, em português e inglês. Este comando também gera
% automaticamente a referência para o próprio documento, conforme as normas
% sugeridas da USP
\begin{resumo}{port}
    Compiladores são grandes programas destinados a tradução de códigos fonte
    entre linguagens de programação distintas, e atualmente estão sendo
    paralelizados para melhor utilizar os recursos dos processadores
    \textit{manycore} através de técnicas de compilação como o \textit{Link
    Time Optimization} (LTO).  Entretanto tal técnica costuma desacelerar o
    processo de desenvolvimento incremental de programas de computador, e pode
    gerar código menos eficiente que o processo clássico de compilação em
    alguns casos. Esse trabalho apresenta uma visão geral sobre o estado da
    arte de paralelismo em compiladores, e propõe uma alternativa ao LTO explorando
    paralelismo interno em um compilador, com \textit{threads}. Para validar
    os resultados, algumas das técnicas discutidas são implementadas no GCC, e
    são realizadas análises no tempo total de compilação do projeto GCC e
    arquivos separados através de técnicas de inferência estatística.
\end{resumo}

% O resumo é obrigatório, em português e inglês. Este comando também gera
% automaticamente a referência para o próprio documento, conforme as normas
% sugeridas da USP
\begin{resumo}{eng}
Compilers are huge software destinated to translate the source code between
distinct programming languages, and currently, they are being parallelized to
better use multicore resources using compiling techniques such as Link Time
Optimization (LTO). This technique usually slows down the process of
interactively software development, and can sometimes generate less efficient
code when compared to the classical compilation process. This thesis presents a
general vision of the state of art about parallelism in compilers and proposes
an alternative to LTO by better exploring parallelism in compiler internals,
through threads. To
validate the obtained results, some of the techniques discussed here are
implemented in GCC, and analyses are performed in the total compilation time of
the GCC project and separated files using statistical inference techniques.
\end{resumo}
