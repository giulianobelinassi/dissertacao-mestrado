%%%%%%%%%%%%%%%%%%%%%%%%%%%%%%%%%%%%%%%%%%%%%%%%%%%%%%%%%%%%%%%%%%%%%%%%%%%%%%%%
%%%%%%%%%%%%%%%%%%%%%%%%%%%%% METADADOS DA TESE %%%%%%%%%%%%%%%%%%%%%%%%%%%%%%%%
%%%%%%%%%%%%%%%%%%%%%%%%%%%%%%%%%%%%%%%%%%%%%%%%%%%%%%%%%%%%%%%%%%%%%%%%%%%%%%%%

% Define o texto da capa e da referência que vai na página do resumo;
% "masc" ou "fem" definem se serão usadas palavras no masculino ou feminino
% (Mestre/Mestra, Doutor/Doutora, candidato/candidata). O segundo parâmetro
% é opcional e determina que se trata de exame de qualificação.
\mestrado[fem]
%\mestrado[fem][quali]
%\doutorado[masc]
%\doutorado[masc][quali]

% Se "\title" está em inglês, você pode definir o título em português aqui
\tituloport{Título do trabalho}

% Se "\title" está em português, você pode definir o título em inglês aqui
\tituloeng{Title of the document}

% Se o trabalho não tiver subtítulo, basta remover isto.
\subtitulo{um subtítulo}

% Se isto não for definido, "\subtitulo" é utilizado no lugar
\subtituloeng{a subtitle}

\orientador[fem]{Profª. Drª. Fulana de Tal}

% Se isto não for definido, "\orientador" é utilizado no lugar
\orientadoreng{Prof. Dr. Fulana de Tal}

% Se não houver, remova
\coorientador[masc]{Prof. Dr. Ciclano}

% Se isto não for definido, "\coorientador" é utilizado no lugar
%\coorientadoreng{Prof. Dr. Ciclano}

\programa{Ciência da Computação}

% Se isto não for definido, "\programa" é utilizado no lugar
\programaeng{Computer Science}

% Se não houver, remova
\apoio{Durante o desenvolvimento deste trabalho o autor recebeu auxílio
financeiro da XXXX}

% Se isto não for definido, "\apoio" é utilizado no lugar
\apoioeng{During this work, the author was supported by XXX}

\localdefesa{São Paulo}

\datadefesa{10 de agosto de 2017}

% Se isto não for definido, "\datadefesa" é utilizado no lugar
\datadefesaeng{August 10th, 2017}

% Necessário para criar a referência do documento que aparece
% na página do resumo
\ano{2017}

\banca{
  \begin{itemize}
    \item Profª. Drª. Nome Completo (orientadora) - IME-USP [sem ponto final]
    \item Prof. Dr. Nome Completo - IME-USP [sem ponto final]
    \item Prof. Dr. Nome Completo - IMPA [sem ponto final]
  \end{itemize}
}

% Se isto não for definido, "\banca" é utilizado no lugar
\bancaeng{
  \begin{itemize}
    \item Prof. Dr. Nome Completo (advisor) - IME-USP [sem ponto final]
    \item Prof. Dr. Nome Completo - IME-USP [sem ponto final]
    \item Prof. Dr. Nome Completo - IMPA [sem ponto final]
  \end{itemize}
}

% Palavras-chave separadas por ponto e finalizadas também com ponto.
\palavraschave{Palavra-chave1. Palavra-chave2. Palavra-chave3.}

\keywords{Keyword1. Keyword2. Keyword3.}

% Se quiser estabelecer regras diferentes, converse com seu
% orientador
\direitos{Autorizo a reprodução e divulgação total ou parcial
deste trabalho, por qualquer meio convencional ou
eletrônico, para fins de estudo e pesquisa, desde que
citada a fonte.}

% Isto deve ser preparado em conjunto com o bibliotecário
%\fichacatalografica{
% nome do autor, título, etc.
%}

%%%%%%%%%%%%%%%%%%%%%%%%%%% CAPA E FOLHAS DE ROSTO %%%%%%%%%%%%%%%%%%%%%%%%%%%%%

% Embora as páginas iniciais *pareçam* não ter numeração, a numeração existe,
% só não é impressa. O comando \mainmatter (mais abaixo) reinicia a contagem
% de páginas e elas passam a ser impressas. Isso significa que existem duas
% páginas com o número "1": a capa e a página do primeiro capítulo. O pacote
% hyperref não lida bem com essa situação. Assim, vamos desabilitar hyperlinks
% para números de páginas no início do documento e reabilitar mais adiante.
\hypersetup{pageanchor=false}

% A capa; o parâmetro pode ser "port" ou "eng" para definir a língua
\capaime[port]
%\capaime[eng]

% Se você não quiser usar a capa padrão, você pode criar uma outra
% capa manualmente ou em um programa diferente. No segundo caso, é só
% importar a capa como uma página adicional usando o pacote pdfpages.
%\includepdf{./arquivo_da_capa.pdf}

% A página de rosto da versão para depósito (ou seja, a versão final
% antes da defesa) deve ser diferente da página de rosto da versão
% definitiva (ou seja, a versão final após a incorporação das sugestões
% da banca). Os parâmetros podem ser "port/eng" para a língua e
% "provisoria/definitiva" para o tipo de página de rosto.
%\pagrostoime[port]{definitiva}
\pagrostoime[port]{provisoria}
%\pagrostoime[eng]{definitiva}
%\pagrostoime[eng]{provisoria}

%%%%%%%%%%%%%%%%%%%% DEDICATÓRIA, RESUMO, AGRADECIMENTOS %%%%%%%%%%%%%%%%%%%%%%%

% A definição deste ambiente está no pacote imeusp; se você não
% carregar esse pacote, precisa cuidar desta página manualmente.
\begin{dedicatoria}
Esta seção é opcional e fica numa página separada; ela pode ser usada para
uma dedicatória ou epígrafe.
\end{dedicatoria}

% Após a capa e as páginas de rosto, começamos a numerar as páginas; com isso,
% podemos também reabilitar links para números de páginas no pacote hyperref.
% Isso porque, embora contagem de páginas aqui começe em 1 e no primeiro
% capítulo também, o fato de uma numeração usar algarismos romanos e a outra
% algarismos arábicos é suficiente para evitar problemas.
\pagenumbering{roman}
\hypersetup{pageanchor=true}

% Agradecimentos:
% Se o candidato não quer fazer agradecimentos, deve simplesmente eliminar
% esta página. A epígrafe, obviamente, é opcional; é possível colocar
% epígrafes em todos os capítulos. O comando "\chapter*" faz esta seção
% não ser incluída no sumário.
\chapter*{Agradecimentos}
\epigrafe{Do. Or do not. There is no try.}{Mestre Yoda}

Texto texto texto texto texto texto texto texto texto texto texto texto texto
texto texto texto texto texto texto texto texto texto texto texto texto texto
texto texto texto texto texto texto texto texto texto texto texto texto texto
texto texto texto texto. Texto opcional.

%% O resumo é obrigatório, em português e inglês. Este comando também gera
% automaticamente a referência para o próprio documento, conforme as normas
% sugeridas da USP
\begin{resumo}{port}
    Compiladores são grandes programas destinados a tradução de códigos fonte
    entre linguagens de programação distintas, e atualmente estão sendo
    paralelizados para melhor utilizar os recursos dos processadores
    \textit{manycore} através de técnicas de compilação como o \textit{Link
    Time Optimization} (LTO).  Entretanto tal técnica costuma desacelerar o
    processo de desenvolvimento incremental de programas de computador, e pode
    gerar código menos eficiente que o processo clássico de compilação em
    alguns casos. Esse trabalho apresenta uma visão geral sobre o estado da
    arte de paralelismo em compiladores, propõe uma alternativa ao LTO,
    explorando melhor o paralelismo no processo clássico de compilação através
    de paralelismo interno em um compilador, e discute técnicas de que podem
    ser empregadas para o paralelismo em compiladores no geral.  Para validar
    os resultados, algumas das técnicas discutidas são implementadas no GCC, e
    são realizadas análises no tempo total de compilação do projeto GCC e
    arquivos separados através de técnicas de inferência estatística.
\end{resumo}

% O resumo é obrigatório, em português e inglês. Este comando também gera
% automaticamente a referência para o próprio documento, conforme as normas
% sugeridas da USP
\begin{resumo}{eng}
Compilers are huge software destinated to translate the source code between
distinct programming languages, and currently they are being parallelized
to better use multicore resources using compiling techniques such as (Link
Time Optimization) LTO. This technique usually slows down the process of
interactively software development, and can sometimes generate less
efficient code when compared to the classical compilation process. This
thesis presents a general vision of the state of art about parallelism in compilers
and proposes an alternative to LTO by better exploring the parallelism
in the classical compilation method by parallelizing the compiler
internals, with a discussion of techniques that can be deployed to
parallelize compilers in general. To validate the obtained results, some of
the techniques discussed here are implemented in GCC, and analyses are
performed in the total compilation time of the GCC project and separated
files using statistical inference techniques.
\end{resumo}

% O resumo é obrigatório, em português e inglês. Este comando também gera
% automaticamente a referência para o próprio documento, conforme as normas
% sugeridas da USP
\begin{resumo}{port}
    Compiladores são grandes programas destinados a tradução de códigos fonte
    entre linguagens de programação distintas, e atualmente estão sendo
    paralelizados para melhor utilizar os recursos dos processadores
    \textit{manycore} através de técnicas de compilação como o \textit{Link
    Time Optimization} (LTO).  Entretanto tal técnica costuma desacelerar o
    processo de desenvolvimento incremental de programas de computador, e pode
    gerar código menos eficiente que o processo clássico de compilação em
    alguns casos. Esse trabalho apresenta uma visão geral sobre o estado da
    arte de paralelismo em compiladores, propõe uma alternativa ao LTO,
    explorando melhor o paralelismo no processo clássico de compilação através
    de paralelismo interno em um compilador, e discute técnicas de que podem
    ser empregadas para o paralelismo em compiladores no geral.  Para validar
    os resultados, algumas das técnicas discutidas são implementadas no GCC, e
    são realizadas análises no tempo total de compilação do projeto GCC e
    arquivos separados através de técnicas de inferência estatística.
\end{resumo}

% O resumo é obrigatório, em português e inglês. Este comando também gera
% automaticamente a referência para o próprio documento, conforme as normas
% sugeridas da USP
\begin{resumo}{eng}
Compilers are huge software destinated to translate the source code between
distinct programming languages, and currently they are being parallelized
to better use multicore resources using compiling techniques such as (Link
Time Optimization) LTO. This technique usually slows down the process of
interactively software development, and can sometimes generate less
efficient code when compared to the classical compilation process. This
thesis presents a general vision of the state of art about parallelism in compilers
and proposes an alternative to LTO by better exploring the parallelism
in the classical compilation method by parallelizing the compiler
internals, with a discussion of techniques that can be deployed to
parallelize compilers in general. To validate the obtained results, some of
the techniques discussed here are implemented in GCC, and analyses are
performed in the total compilation time of the GCC project and separated
files using statistical inference techniques.
\end{resumo}



%%%%%%%%%%%%%%%%%%%%%%%%%%% LISTAS DE FIGURAS ETC. %%%%%%%%%%%%%%%%%%%%%%%%%%%%%

% Como as listas que se seguem podem não incluir uma quebra de página
% obrigatória, inserimos uma quebra manualmente aqui.
\makeatletter
\if@openright\cleardoublepage\else\clearpage\fi
\makeatother

% Todas as listas são opcionais; Usando "\chapter*" elas não são incluídas
% no sumário. As listas geradas automaticamente também não são incluídas
% por conta das opções "notlot" e "notlof" que usamos mais acima.

% Normalmente, "\chapter*" faz o novo capítulo iniciar em uma nova página, e as
% listas geradas automaticamente também por padrão ficam em páginas separadas.
% Como cada uma destas listas é muito curta, não faz muito sentido fazer isso
% aqui, então usamos este comando para desabilitar essas quebras de página.
% Se você preferir, comente as linhas com esse comando e des-comente as linhas
% sem ele para criar as listas em páginas separadas. Observe que você também
% pode inserir quebras de página manualmente (com \clearpage, veja o exemplo
% mais abaixo).
\newcommand\disablenewpage[1]{{\let\clearpage\par\let\cleardoublepage\par #1}}

% Nestas listas, é melhor usar "raggedbottom" (veja basics.tex). Colocamos
% a opção correspondente e as listas dentro de um par de chaves para ativar
% raggedbottom apenas temporariamente.
{
\raggedbottom

%%%%% Listas criadas manualmente

%\chapter*{Lista de Abreviaturas}
\disablenewpage{\chapter*{Lista de Abreviaturas}}

\begin{tabular}{rl}
    GCC          & Gnu Compiler Collection\\
	URL          & Localizador Uniforme de Recursos (\emph{Uniform Resource Locator})\\
	IME          & Instituto de Matemática e Estatística\\
	USP          & Universidade de São Paulo
\end{tabular}

%\chapter*{Lista de Símbolos}
\disablenewpage{\chapter*{Lista de Símbolos}}

\begin{tabular}{rl}
        $\omega$    & Frequência angular\\
        $\psi$      & Função de análise \emph{wavelet}\\
        $\Psi$      & Transformada de Fourier de $\psi$\\
\end{tabular}

% Quebra de página manual
\clearpage

%%%%% Listas criadas automaticamente

%\listoffigures
\disablenewpage{\listoffigures}

%\listoftables
\disablenewpage{\listoftables}

% Esta lista é criada "automaticamente" pela package float quando
% definimos o novo tipo de float "program" (em utils.tex)
%\listof{program}{\programlistname}
\disablenewpage{\listof{program}{\programlistname}}

% Sumário (obrigatório)
\tableofcontents

} % Final de "raggedbottom"

% Referências indiretas ("x", veja "y") para o índice remissivo (opcionais,
% pois o índice é opcional). É comum colocar esses itens no final do documento,
% junto com o comando \printindex, mas em alguns casos isso torna necessário
% executar texindy (ou makeindex) mais de uma vez, então colocar aqui é melhor.
\index{Inglês|see{Língua estrangeira}}
\index{Figuras|see{Floats}}
\index{Tabelas|see{Floats}}
\index{Código-fonte|see{Floats}}
\index{Subcaptions|see{Subfiguras}}
\index{Sublegendas|see{Subfiguras}}
\index{Equações|see{Modo Matemático}}
\index{Fórmulas|see{Modo Matemático}}
\index{Rodapé, notas|see{Notas de rodapé}}
\index{Captions|see{Legendas}}
\index{Versão original|see{Tese/Dissertação, versões}}
\index{Versão corrigida|see{Tese/Dissertação, versões}}
\index{Palavras estrangeiras|see{Língua estrangeira}}
\index{Floats!Algoritmo|see{Floats, Ordem}}
