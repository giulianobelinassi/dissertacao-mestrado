% Vamos definir alguns comandos auxiliares para facilitar.

% "textbackslash" é muito comprido.
\newcommand{\sla}{\textbackslash}

% Vamos escrever comandos (como "make" ou "itemize") com formatação especial.
\newcommand{\cmd}[1]{\textsf{#1}}

% Idem para packages; aqui estamos usando a mesma formatação de \cmd,
% mas poderíamos escolher outra.
\newcommand{\pkg}[1]{\textsf{#1}}

% A maioria dos comandos LaTeX começa com "\"; vamos criar um
% comando que já coloca essa barra e formata com "\cmd".
\newcommand{\ltxcmd}[1]{\cmd{\sla{}#1}}

\chapter{Related Works}
\label{chap:related_works}

A large portion of the research in compilers regarding parallelism is allocated
to automatically parallelizing the input code. There are two areas of study
with are related to compiler parallelization, which is parallel parsing and
compiling \textit{de facto}. We present the works in this area separately.

%Dos algoritmos apresentados na Seção \ref{chap:fundamentacao}, o algoritmo
%Recursivo Descendente é o mais simples deles. \cite{Lincoln:1970:PPT:987475.987478}
%propôs a paralelização da análise léxica de um compilador
%Fortran por meio da quebra do código fonte em partes menores. Para isso, o algoritmo proposto mapeia
%informações de final de
%linha e de posicionamento de cada caractere e realiza um processamento através de operações em APL.
%Posteriormente, \cite{Krohn:1975:PAC:390015.808414} estendeu o trabalho
%anterior, propondo uma maneira de realizar a análise sintática, a
%tradução para a AST, e a geração
%de código utilizando os registradores vetoriais do supercomputador
%STAR-100. Ambos os artigos não apresentam nenhuma
%análise assintótica para os algoritmos propostos, e tampouco apresentam
%experimentos.

\begin{section}{Parallel Parsing}

From the parsing algorithms presented in Section \ref{chap:fundamentacao}, the
recursive descent parser is the simplest of them.
\cite{Lincoln:1970:PPT:987475.987478} proposed a parallel lexical analysis on a
Fortran compiler by dividing the input source code into smaller parts, and maps
the position of each character and linefeed in the code, and process them
through APL operations. Further, \cite{Krohn:1975:PAC:390015.808414} extended
this work for syntactic analysis and code generation, using the vectorial units
of the (so far) STAR-100 supercomputer. Both papers do not present experiments
nor asymptotic analysis of this algorithm.

\cite{fischer1975parsing} gives a detailed theoretical study, proving several
concurrent parsing techniques for the LR($k$) family. They propose a parallel
LR($k$) by breaking the input into several arbitrary parts, and running a
serial parser on each of them. Then the algorithm tries to recover the stack of
each noninitial parser by constructing a set of possible states, for which
there are 5 possible cases. However, in case of an error, the parser result
should be discarded, and therefore a lot of work will be done in vain when
comparing with the sequential version.

Furthermore, \cite{Mickunas:1978:PCM:800127.804105} also proposed a parallelization
of the LR($0$) algorithm. His idea consists of breaking the input arbitrarily
points and executing a serial parser for each segment, which he names Piecewise
LR parsing. These parsers (except the first) are modified to avoid conflict by
adding an "initial" state by adding every $A \rightarrow u \bullet v$, where $A \rightarrow uv \in
R$ and $v \neq \lambda$, to the parser item set. This process may add undesired
states which creates reduce-reduce or shift-reduce conflicts, which the author
provides a heuristic for solving them.  If the parser finds a conflict which
this heuristic cannot solve, then it sends the stack content to the
left-neighborhood, and that parser resumes the processing. The authors
do not show speedup experiments nor assyntotic analysis of this method.
\cite{Pennello:1978:FMA:512760.512786} implemented this strategy in a Pascal
compiler.

\cite{fowler2009parallel} described how to parallelize Earley
and Packrat methods. Although they are mostly used for natural language
processing, these algorithms can be used to parse computer languages as well.
For the Earley algorithm, the authors partition the Earley sets into
sub-blocks and run each block in parallel. For solving the dependency across
the blocks, the authors propose a way to speculate additional items into the
parser, which are not produced by the serial algorithm. For Packrat, the authors
propose a message passing mechanism.  The input is divided into parts, and each
part is assigned to a worker thread. Each thread speculates until the
thread on its left has finished parsing, and send the synthesized starting
symbol to the thread on its right. The authors managed a speedup of $5.5\times$
in Earley, and $2.5\times$ in Packrat.

Still on Packrat methods, \cite{dubroy2017incremental} show
how to implement an incremental Pacrkat to avoids reparsing the entire input
on modifications. The authors achieve this by recording all its
intermediate results in a parsing table and modify the parsing program to reload
this table when invoked. The authors showed that their method does not require
any modification to the original grammar, and their method requires only
up to $11.7\%$ of extra memory when compared to the original parser. Authors
claim a reduction from $23.7ms$ to $6.2ms$ on reparsing a 279Kb input string.

\cite{Barenghi:2015:PPM:2839536.2840146} explore some properties of Operator
Precedence Grammars to construct a Yacc-like parser constructor named PAPAGENO,
which generates parallel parsers. Operator Precedence Grammars are Context-Free
Grammars with an operator precedence table, which given two symbols $a, b \in
\Si$, instruct which symbols should take or yield precedence. The authors
described precedence grammars for Lua and JSON, which they used in their tests
to get a speedup of up to $5.5\times$ when compared to a parser generated by
GNU Bison.

\end{section}

%Posteriormente, \cite{Mickunas:1978:PCM:800127.804105} propôs uma
%paralelização do algoritmo LR($0$). Sua ideia consiste em quebrar
%arbitrariamente a entrada em vários segmentos e executar um analisador em paralelo
%em cada um destes segmentos. Estes analisadores (exceto o primeiro,
%que executa no início da entrada) são modificados na tentativa de
%evitar conflitos, uma vez que modificar a ordem de início da análise
%pode impossibilitar a distinção entre o prefixo e o sufixo de uma fase.
%Quando um conflito é detectado, o analisador envia a sequência de
%\textit{tokens} lida até então ao analisador à sua esquerda,
%retorna ao estado inicial, e reinicia a sua análise.
%No termino da análise, cada analisador envia o resultado para o analisador
%a sua esquerda, acumulando o resultando no primeiro.
%Os autores não mostraram experimentos ou
%análise assintótica do algoritmo proposto. \cite{Pennello:1978:FMA:512760.512786}
%implementou essa estratégia em um compilador Pascal, mas seus experimentos foram
%realizados em um computador paralelo simulado.

\begin{section}{Parallel Compilation}

 \cite{vandevoorde1988parallel} parallelized the \textit{Titan C Compiler}, A C
 compiler written in Modula-2. His implementation consists of a syntactic
 analyzer running in parallel to the lexical analyzer, which feeds the tokens
 through a pipeline. The author uses a recursive descent parser that, after
 reading the function declarations, executes new threads for every statement in
 the program. For this, the author assumes that every function is declared in
 the source header. This fine granularity strategy required a way to control
 the parallelism, which the authors used the concept of workcrews that limits
 the number of threads executing simultaneously. The author reports a speedup
 of $10\%$ when using this pipeline between the analyzers, and a speedup of up
 to $3.1\times$ when using 5 MicroVAX II processors in shared memory. This
 compiler does not include optimizations, and therefore it is not discussed in
 the paper.


%\cite{vandevoorde1988parallel} paralelizou o \textit{Titan C compiler}, um compilador
%C escrito em Modula-2. Sua implementação consiste na execução de um analisador léxico
%em paralelo a um analisador sintático, sendo os \textit{tokens} alimentados
%através de um \textit{pipeline}. O autor também propôs a paralelização de um
%analisador sintático recursivo descendente que, após a sequência de declarações do programa, executa novas \textit{threads}
%para cada expressão da gramática e tem
%como premissa que qualquer declaração de funções do programa está no
%cabeçalho do
%arquivo. Essa estratégia de granularidade fina necessitou a implementação de
%uma maneira de controlar o paralelismo, onde foi utilizado o conceito de
%WorkCrews \citep{vandevoorde1988workcrews} que limita o número máximo de
%\textit{threads} em execução simultânea. O autor relata um ganho de 10\% no
%uso do \textit{pipeline} entre o analisador léxico e sintático, e um
%\textit{speedup} de até $3.1\times$ utilizando 5 processadores MicroVAX II em
%memória compartilhada. Entretanto, tal compilador não possui estágios de
%otimização, e assim, métodos de paralelização de um otimizador não foram discutidos neste trabalho.

Similarly, \cite{wortman1992} implemented a parallel Modula-2+ compiler. Their strategy
consisted of using a lexical analyzer capable of detecting functions in the
code and proceeding with parallel compilation at this point. To solve problems
with function calls and access to symbols yet not defined, the authors propose
a ternary logic in the symbol table, with a state ``\textit{doesn't know yet}'' and
propose three strategies to implement this functionality. The authors also used
a fixed number of worker threads and a priority producer-consumer queue on the
syntactic analyzer to the code generator, where the worker threads will pull
jobs from. This priority queue is to process larger functions first.
Experiments conducted by the authors showed speedups ranging from $1.5\times$ to
$6\times$ times using 8 MicroVAX II in shared memory.


%De maneira similar, \cite{wortman1992} implementou um compilador de Modula-2+
%paralelo. A estratégia consiste em utilizar um analisador léxico capaz de
%encontrar funções no código fonte e prosseguir com a compilação em paralelo
%nesse nível, gerando uma tarefa para cada função. Para solucionar problemas
%relacionados à chamada de funções e
%acesso a símbolos ainda não definidos, o autor propõe o uso de
%uma lógica ternária na tabela de símbolos, especificando um estado
%``\textit{Doesn't know yet}'' (Ainda não visto), e propõe três estratégias para
%implementar essa funcionalidade. Em seguida, o autor propõe o uso de
%um número fixo de \textit{threads} trabalhadoras, e também o uso de uma fila de
%prioridade produtor consumidor, onde as \textit{threads} deverão retirar o trabalho.
%O uso de uma fila de prioridade permite que as funções mais longas sejam
%processadas primeiro. Os experimentos conduzidos pelo autor mostraram
%\textit{speedups} variando de $1.5\times$ até $6\times$ utilizando 8
%processadores MicroVAX II em memória compartilhada.

%Os artigos apresentados até então não abordam questões relacionadas
%a otimização de código. Isso porque os compiladores naquela época possuíam poucos
%passos de otimização, o que implicava em pouco tempo necessário para fazê-la.
%Mesmo assim, alguns pesquisadores dedicaram-se a estudar esses passos.
%\cite{Lee1994} estudaram paralelismo na análise de controle
%de fluxo. Como justificativa para esse estudo, os autores afirmam que a maior parte
%do tempo gasto por compiladores otimizadores é com este tipo de análise,
%principalmente nas otimizações Inter Procedurais. 
%A investigação se concentrou em algoritmos que usam o grafo de controle
%de fluxo como entrada. Os autores propõem uma heurística de aglutinação
%para particionar tal grafo em regiões conexas e adjacentes de tamanho não maior
%que um certo inteiro $s$. O uso da heurística é por razão do problema de
%de particionamento ser NP-difícil.
%Os experimentos da implementação utilizaram
%8 processadores do computador iPSC/2 em memória distribuída, e 
%obtiveram \textit{speedups} variando de
%$2.8 \times$ até $6.5\times$. Os autores argumentaram que o \textit{speedup} foi
%limitado devido às características dos programas no teste: os arquivos consistiam
%de várias funções pequenas, o que normalmente é considerado uma boa prática de
%programação.

The papers presented so far do not touch questions regarding optimizations.
This is because compilers so far had none to few optimization passes, which
implied in little time required to process them. Yet, some researchers studied
how to parallelize these passes. \cite{Lee1994} studied parallelism in both
intraprocedural and interprocedural dataflow analysis. The authors justify this
study by stating that the most compilation time is spent in this part. The
authors proposed a hybrid algorithm, where the CFGraph is decomposed in
strongly connected partitions. Each partition runs its dataflow analysis
independently, then this local result is propagated to the other partitions,
and finally, this result is propagated again inside the nodes of other
partitions. Authors report a speedup ranging from $1.7\times$ to $7.8\times$
regarding the sequential algorithm.

Another way of exploring parallelism is by converting the CFGraph into a DAG.
\cite{kramer1994combining} showed that converting any CFGraph to a DAG
structure still yields the correct dataflow solution of the original problem
while removing any backedge from the original CFGraph. Then the authors show
how to construct a dependency graph from this DAG, which they process in
parallel by levels. Authors show a speedup from $2.7\times$ to $5.4\times$.

%Uma outra forma de explorar o paralelismo utilizando um Grafo Dirigido Acíclico
%foi proposta por \cite{kramer1994combining}. Os autores mostram formas de propagar
%em paralelo as informações nesse grafo explorando seus caminhos independentes,
%e mostra uma maneira de transformar um Grafo de Controle de Fluxo com laços em
%um Grafo de Controle de Fluxos acíclico equivalente. Os experimentos realizados
%pelos autores apenas mostram oportunidades de paralelismo encontrada pelo
%algoritmo.

\begin{subsection}{Link Time Optimization (LTO)}\label{lto_section}

The issue around this scheme is that it can only optimize with
the information found in its TU because it can not see the body
content of other files' functions. A TU is the
entire content of a source file (a .c file in C) plus all its headers.

As an answer to this, LTO allows cross-module optimizations by
postponing optimizations and final translation to a linker wrapper. There, the entire
program can be loaded by the compiler (but more often, just some sort of summary)
as a single, big TU, and optimizations can be decided globally,
as now it has access to the internals of other modules. LTO is divided into
three steps \citep{whoprgoogle,glek2010optimizing}:
\begin{itemize}
\item LGEN (\textit{Local Generation}): each module is translated to an IR and
written to disk in phony object files. These objects are \emph{phony} because
they do not contain assembly code. This step runs serially on the input file
(\textit{i.e.} in parallel to the files in the project).

\item WPA (\textit{Whole Program Analysis}): load all translated modules,
merges all TUs into one, and analyzes the program globally. After that, it
generates an \emph{optimization summary} for the program and partitioned this
global TU for the next stage. This analysis runs sequentially to the entire
project.

\item LTRANS (\textit{Local Transformations}): apply the transformations generated by
WPA to each partition, which will generate its own object file. This stage runs in
parallel.
\end{itemize}

This process is sketched in Fig. \ref{fig:whopr_build}, where the linker
wrapper is represented by \textit{collect2}, which firstly launch \textit{lto1}
in WPA mode, and the second time it finally launches \textit{ld}. This process
can be seen by launching gcc with \texttt{-flto -v}.

\begin{figure}
\tikzstyle{block} = [rectangle, draw, fill=white,
    text width=6em, text centered, rounded corners, node distance=1cm and 0.5cm, minimum height=2em]
\tikzstyle{line} = [draw, -latex]
\makebox[\textwidth][c]{

\scalebox{0.8}{
\begin{tikzpicture}[node distance = 3cm, auto]
    % Place nodes
    \node [block]              (fonte1) {source1.c};
    \node [block, right= of fonte1]        (fonte2) {source2.cpp};
    \node [block, right= of fonte2]        (fonte3) {source3.f90};
    \node [block, above= of fonte2]         (make)   {Makefile};

    \node [block, below= of fonte1]        (gcc)      {gcc};
    \node [block, below= of fonte2]        (g++)      {g++};
    \node [block, below= of fonte3]        (gfortran) {gfortran};

    \node [block, below= of gcc]           (objeto1) {obj1.o};
    \node [block, below= of g++]           (objeto2) {obj2.o};
    \node [block, below= of gfortran]      (objeto3) {obj3.o};

    \node [block, below= of objeto2]       (gcc_lto) {collect2 (lto1)};

    \node [block, right= of fonte3]            (gcc_ltrans1) {gcc\_ltrans};
    \node [block, right= of gcc_ltrans1]   (gcc_ltrans2) {gcc\_ltrans};
    \node [block, right= of gcc_ltrans2]   (gcc_ltrans3) {gcc\_ltrans};

    \node [block, above= of gcc_ltrans2]       (gcc_wpa) {gcc\_wpa};
    \coordinate[below= of gcc_wpa]            (c2);


    \node [block, below= of gcc_ltrans1]   (obj1) {obj1.o};
    \node [block, below= of gcc_ltrans2]   (obj2) {obj2.o};
    \node [block, below= of gcc_ltrans3]   (obj3) {obj3.o};

    \node [block, below=of obj2]   (ld) {collect2 (LD)};

	\node [block, below=of ld]   (bin) {Binary};

    % Draw edges
    \draw[->]    ([xshift=-0.7em] make.south)   -- (fonte1.north);
    \draw[->]    (make.south)   -- (fonte2.north);
    \draw[->]    ([xshift=+0.7em] make.south)   -- (fonte3.north);

    \draw[->]    (fonte1.south)   -- (gcc.north);
    \draw[->]    (fonte2.south)   -- (g++.north);
    \draw[->]    (fonte3.south)   -- (gfortran.north);

    \draw[->]    (gcc.south)   -- (objeto1.north);
    \draw[->]    (g++.south)   -- (objeto2.north);
    \draw[->]    (gfortran.south)   -- (objeto3.north);

    \draw[->]    (objeto1.south)   -- ([xshift=-0.7em]gcc_lto.north);
    \draw[->]    (objeto2.south)   -- (gcc_lto.north);
    \draw[->]    (objeto3.south)   -- ([xshift=+0.7em]gcc_lto.north);

    %\draw[->]    (gcc_lto.south)   -- (gcc_wpa.north);
 	\draw[->]  (gcc_lto.east) .. controls +(7.5,0) and +(-7.5,0).. (gcc_wpa.west);

    \draw[->]    (gcc_wpa.south)   -- (gcc_ltrans1.north);
    \draw[->]    (gcc_wpa.south)   -- (gcc_ltrans2.north);
    \draw[->]    (gcc_wpa.south)   -- (gcc_ltrans3.north);

    \draw[->]    (gcc_ltrans1.south)   -- (obj1.north);
    \draw[->]    (gcc_ltrans2.south)   -- (obj2.north);
    \draw[->]    (gcc_ltrans3.south)   -- (obj3.north);

    \draw[->]    (obj1.south)   -- ([xshift=-0.7em]ld.north);
    \draw[->]    (obj2.south)   -- (ld.north);
    \draw[->]    (obj3.south)   -- ([xshift=+0.7em]ld.north);

	\draw[->]    (ld.south)   -- (bin.north);
	
	%draw brackets
\draw [decorate,decoration={brace,amplitude=10pt},xshift=-0.5cm,yshift=0pt]
([xshift=-1.3cm]objeto1.south) -- ([xshift=-1.3cm]fonte1.north) node [black,midway,xshift=-0.3cm]
{\footnotesize \begin{turn}{90}LGEN\end{turn}};

\draw [decorate,decoration={brace,amplitude=10pt},xshift=-0.5cm,yshift=0pt]
([xshift=1.3cm]gcc_ltrans3.north) -- ([xshift=1.3cm]obj3.south) node [black,midway,xshift=0.3cm]
{\footnotesize \begin{turn}{-90}LTRANS\end{turn}};


\draw [decorate,decoration={brace,amplitude=10pt},xshift=-1.0cm,yshift=0pt]
([xshift=1.3cm]gcc_wpa.north) -- ([xshift=1.3cm]gcc_wpa.south) node [black,midway,xshift=0.3cm]
{\footnotesize \begin{turn}{-90}WPA\end{turn}};


\end{tikzpicture}
}
}%
\caption{Compilation of a program using LTO scheme}
\label{fig:whopr_build}
\end{figure}

\cite{glek2010optimizing} report that LTO produced faster and smaller binaries when compiling GCC
and Firefox, however with a two times slower compilation in GCC. Interestingly,
Firefox compiled 20minutes faster. In all reported cases, the intraprocedural
optimizations dominated the compilation time.

\cite{livska2014optimizing} complements the analysis of this functionality. One
of the problems found was the high memory consumption of LTO, which went up to
12Gb when compiling Chromium, and 15Gb when compiling Linux. The biggest
criticism was the necessary recompilation time when a single file is modified
in LTO mode when compared with the classical compilation scheme.

There are also papers using a profile-based approach rather than a whole-program
analysis for cross-module optimizations.  Xinliang \textit{et al.}
\cite{lipo} proposed LIPO, which embeds IPA in the generated
binary with minimal overhead, and loads information about functions in other
modules based on the profiling report. LIPO also removes the requirement of
phony object dumps and link time wrappers when compared to LTO. Results showed
a speedup of up to $10\times$ in compilation time when compared to Feedback
Driven Optimization (FDO) alone.

Following that, Johnson \textit{et al.} proposes ThinLTO \cite{thinlto}. It
differs from LTO by generating the function summaries on the LGEN stage for
improved parallelism, while avoiding the large memory footprint requirement on the
original LTO. ThinLTO also supports incremental builds by hashing the compiled
function to avoid recompilation of unmodified functions, while LTO requires
recompilation of the entire partition. Today, most GCC's IPA
optimizations only use summaries to avoid this memory footprint issue.

\end{subsection}
\end{section}

%Após estes resultados, houve um aparente hiato nas pesquisas relacionadas a
%compilação em paralelo. As possíveis razões são a complexidade de um compilador
%como programa, e o modelo dos processadores daquela época, que ainda eram
%majoritariamente sequenciais. Uma década depois, houve uma explosão de processadores
%paralelos e do poder computacional como um todo, como apresentado na Seção
%\ref{sec:parallel_comp}. Este ganho computacional levou às empresas a propor soluções
%ainda mais complexas para
%gerar código mais eficiente. Embora apresentadas como maneiras de otimizar o código
%dado como entrada, há relação nessas soluções com paralelismo em compiladores.

%\cite{whoprgoogle} propuseram uma alteração no
%compilador GCC com a finalidade
%de possibilitar otimizações no programa como um todo. A alteração consiste
%em construir um grafo de chamadas de função em relação a todo o programa,
%ao contrário do modelo clássico, em que cada módulo é compilado independentemente.
%Esse processo é composto por três etapas:
%\begin{enumerate}
%    \item \textit{Local Generation } (LGEN). Cada função do programa é compilada
%        na Linguagem Intermediária, em conjunto com o grafo de chamadas de função.
%        Esse estágio pode ser executado em paralelo.
%
%    \item \textit{Whole Program Analysis} (WPA). O grafo de chamadas de função global
%        é construído e as otimizações a serem efetuadas são selecionadas. Esse estágio
%        deve ser executado sequencialmente.
%
%    \item \textit{Local Transformations} (LTRANS). Todas as decisões aplicadas na
%        etapa 2 são implementadas localmente e o código objeto final é gerado.
%        Esse estágio pode executar em paralelo.
%\end{enumerate}


%A Figura \ref{fig:whopr_build} retrata este processo. Primeiro, cada arquivo é
%compilado para a Linguagem Intermediária do Compilador (no GCC, a GIMPLE), em
%conjunto com o seu Grafo de Chamadas de Função, gerando arquivos objeto
%\texttt{lto\_obj*.o}. Esse passo ocorre em paralelo para cada arquivo.
%Em sequência, o gcc\_wpa é chamado, unindo todos os grafos
%de chamada de função e decidindo como aplicar cada otimização Inter Procedural para
%cada função do programa. Em seguida, o gcc\_ltrans é chamado para cada partição do
%Grafo de Chamadas de Função, gerando os arquivos objetos para que o LD por
%fim gere o binário final.
%
%Essa proposta foi implementada
%no GCC \citep{glek2010optimizing} com o nome de
%\textit{Link Time Optimization} (LTO). A granularidade
%escolhida na etapa LGEN foi paralelismo por arquivo. Já na LTRANS a granularidade
%escolhida foram partições do arquivo objeto, o que possibilitou, como
%consequência, a otimização de funções de um arquivo em paralelo. Os autores
%testaram essa implementação com o
%código fonte do Mozilla Firefox e do GCC, e relataram que o GCC construiu
%binários menores e mais rápidos. Os autores relataram que essa implementação tornou
%o processo de autocompilação do GCC duas vezes mais lenta,
%mas que também houve um ganho de 20 minutos na compilação do Firefox. Em todos os
%casos relatados, os passos de otimização Intra Procedural consumiu mais tempo.
%\cite{livska2014optimizing} completa a análise dessa funcionalidade.
%Um dos problemas encontrados foi o uso elevado de memória para carregar o grafo de
%chamadas de função global necessário na fase WPA, que atingiu 12Gb na compilação do
%Chromium e 15Gb na compilação do Kernel Linux. A maior crítica desse processo
%foi o tempo necessário para recompilar todo o programa quando uma simples alteração
%era efetuada em um único arquivo, o que inviabilizava um desenvolvimento incremental.
%
%\begin{figure}
%\tikzstyle{block} = [rectangle, draw, fill=white,
%    text width=6em, text centered, rounded corners, node distance=3cm, auto, minimum height=2em]
%\tikzstyle{line} = [draw, -latex]
%\tikzstyle{cloud} = [draw, ellipse,fill=white, node distance=2cm,
%    minimum height=2em]
%\begin{center}
%\scalebox{0.7}{
%\begin{tikzpicture}[node distance = 3cm, auto]
%    % Place nodes
%    \node [block]                         (make)   {Makefile};
%    \coordinate[below of=make]            (c);
%    \node [block, left of=c]              (fonte1) {fonte1.c};
%    \node [block, right of=fonte1]        (fonte2) {fonte2.cpp};
%    \node [block, right of=fonte2]        (fonte3) {fonte3.f90};
%
%    \node [block, below of=fonte1]        (gcc)      {gcc};
%    \node [block, below of=fonte2]        (g++)      {g++};
%    \node [block, below of=fonte3]        (gfortran) {gfortran};
%
%    \node [block, below of=gcc]           (objeto1) {lto\_obj1.o};
%    \node [block, below of=g++]           (objeto2) {lto\_obj2.o};
%    \node [block, below of=gfortran]      (objeto3) {lto\_obj3.o};
%
%    \node [block, below of=objeto2]       (gcc_lto) {gcc\_lto1};
%
%    \node [block, below of=gcc_lto]       (gcc_wpa) {gcc\_wpa};
%    \coordinate[below of=gcc_wpa]            (c2);
%
%    \node [block, left of=c2]            (gcc_ltrans1) {gcc\_ltrans};
%    \node [block, right of=gcc_ltrans1]   (gcc_ltrans2) {gcc\_ltrans};
%    \node [block, right of=gcc_ltrans2]   (gcc_ltrans3) {gcc\_ltrans};
%
%    \node [block, below of=gcc_ltrans1]   (obj1) {obj1.o};
%    \node [block, below of=gcc_ltrans2]   (obj2) {obj2.o};
%    \node [block, below of=gcc_ltrans3]   (obj3) {obj3.o};
%
%    \node [block, below of=obj2]   (ld) {LD};
%
%	\node [block, below of=ld]   (bin) {Executável};
%
%    % Draw edges
%    \draw[->]    ([xshift=-0.7em] make.south)   -- (fonte1.north);
%    \draw[->]    (make.south)   -- (fonte2.north);
%    \draw[->]    ([xshift=+0.7em] make.south)   -- (fonte3.north);
%
%    \draw[->]    (fonte1.south)   -- (gcc.north);
%    \draw[->]    (fonte2.south)   -- (g++.north);
%    \draw[->]    (fonte3.south)   -- (gfortran.north);
%
%    \draw[->]    (gcc.south)   -- (objeto1.north);
%    \draw[->]    (g++.south)   -- (objeto2.north);
%    \draw[->]    (gfortran.south)   -- (objeto3.north);
%
%    \draw[->]    (objeto1.south)   -- ([xshift=-0.7em]gcc_lto.north);
%    \draw[->]    (objeto2.south)   -- (gcc_lto.north);
%    \draw[->]    (objeto3.south)   -- ([xshift=+0.7em]gcc_lto.north);
%
%    \draw[->]    (gcc_lto.south)   -- (gcc_wpa.north);
%
%    \draw[->]    (gcc_wpa.south)   -- (gcc_ltrans1.north);
%    \draw[->]    (gcc_wpa.south)   -- (gcc_ltrans2.north);
%    \draw[->]    (gcc_wpa.south)   -- (gcc_ltrans3.north);
%
%    \draw[->]    (gcc_ltrans1.south)   -- (obj1.north);
%    \draw[->]    (gcc_ltrans2.south)   -- (obj2.north);
%    \draw[->]    (gcc_ltrans3.south)   -- (obj3.north);
%
%    \draw[->]    (obj1.south)   -- ([xshift=-0.7em]ld.north);
%    \draw[->]    (obj2.south)   -- (ld.north);
%    \draw[->]    (obj3.south)   -- ([xshift=+0.7em]ld.north);
%
%	\draw[->]    (ld.south)   -- (bin.north);
%
%
%\end{tikzpicture}
%}
%\end{center}
%\caption{Compilação de um programa utilizando WHOPR.}
%\label{fig:whopr_build}
%\end{figure}


%Considerando o alto uso de memória e tempo de compilação nos casos relatados
%acima, \cite{Johnson:2017:TSI:3049832.3049845} propôs o ThinLTO. A ideia é
%considerar apenas parte do grafo global a cada vez, evitando carregá-lo
%completamente na memória; e postergar ao máximo aplicar as otimizações possíveis,
%realizando apenas análises de maneira sequencial e aplicando-as no 
%\textit{backend} em paralelo.
%Os autores relataram um \textit{speedup} de até $15\times$ quando comparado a
%implementação do LTO original do Clang e até $4\times$ quando comparado a
%implementação do LTO do GCC. Os autores também relataram uma diminuição considerável
%no consumo de memória quando comparado ao LTO do GCC: Redução de 2.9Gb para 1Gb
%na compilação do Clang, 10.8Gb para 1.0Gb no Chromium, e permitiu melhores otimizações no
%\textit{Ad Delivery} da Google, enquanto a implementação do LTO no GCC
%abortava a compilação após consumir mais de 25Gb de memória. O desempenho dos
%binários gerados também foi similar, inclusive onde o LTO original gerava
%binários mais lentos que o processo clássico de compilação. O tempo de compilação
%também melhorou significativamente quando comparado ao LTO original,
%mas permaneceu pior em todos os casos quando comparado ao processo clássico de
%compilação.

%Houve também nas pesquisas um renascimento de análise sintática em paralelo.
%\cite{Barenghi:2015:PPM:2839536.2840146} encontrou uma maneira de explorar
%a propriedade de Análise Local em Gramáticas com Precedência de Operadores.
%Isso permitiu a construção de um gerador de analisadores sintáticos
%PAPAGENO (\textit{Parallel Parser Generator}). Para isto, a gramática a
%ser utilizada nesse analisador precisa ser convertida para uma Gramática
%com Precedência de Operadores, conforme discutido e exemplificado pelos
%autores. Testes empíricos foram efetuados implementando um analisador
%JSON e outro analisador para a linguagem Lua. Em um Opteron 8378 (16 núcleos),
%os autores atingiram um \textit{speedup} de até $5.3\times$ no análisador JSON
%em arquivos grandes, e pouco mais de $4\times$ no analisador Lua. O algoritmo
%paralelo foi comparado com os gerados pelo analisador sintático \textit{Bison},
%que apenas gera código sequencial.
%
%Atualmente, os compiladores são \textit{softwares} bem mais complexos que
%aqueles apresentados por \cite{vandevoorde1988workcrews} e \cite{wortman1992},
%havendo muito mais passos de otimização, enquanto houve pouquíssima ou nenhuma
%adição de complexidade nos tópicos de análise léxica, sintática, e geração de
%código o que esses trabalhos procuraram explorar. Já os trabalhos de
%\cite{Lee1994} e \cite{kramer1994combining} são relevantes por mostrar
%estratégias de paralelização de otimizações Inter Procedurais, entretanto elas
%não são a maioria das otimizações implementadas nos compiladores. Por outro
%lado, o LTO e o ThinLTO não foram projetados para serem usadas em
%desenvolvimento incremental, além de gerar binários menos eficientes em alguns
%casos. Efetuando a paralelização do compilador internamente (usando
%\textit{threads}, por exemplo) a estrutura original do compilador é preservada,
%evitando esse problema.
