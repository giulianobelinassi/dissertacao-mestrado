%!TeX root=../tese.tex
%("dica" para o editor de texto: este arquivo é parte de um documento maior)
% para saber mais: https://tex.stackexchange.com/q/78101/183146

%%%%%%%%%%%%%%%%%%%%%%%%%%%%%%%%%%%%%%%%%%%%%%%%%%%%%%%%%%%%%%%%%%%%%%%%%%%%%%%%
%%%%%%%%%%%%%%%%%%%%%%%%%%%%% METADADOS DA TESE %%%%%%%%%%%%%%%%%%%%%%%%%%%%%%%%
%%%%%%%%%%%%%%%%%%%%%%%%%%%%%%%%%%%%%%%%%%%%%%%%%%%%%%%%%%%%%%%%%%%%%%%%%%%%%%%%

% Estes comandos definem o título e autoria do trabalho e devem sempre ser
% definidos, pois além de serem utilizados para criar a capa, também são
% armazenados nos metadados do PDF.
\title{
    % Obrigatório nas duas línguas
    titlept={Paralelizando um Compilador: Um Estudo com o GCC},
    titleen={Parallelizing a Compiler: A Study with GCC},
    % Opcional, mas se houver deve existir nas duas línguas
    %subtitlept={um subtítulo},
    %subtitleen={a subtitle},
    % Opcional, para o cabeçalho das páginas
    shorttitle={Parallelizing a Compiler},
}

\author[masc]{Giuliano Augusto Faulin Belinassi}

% Para TCCs, este comando define o supervisor
\orientador[masc]{Prof. Dr. Alfredo Goldman}

% Se não houver, remova; se houver mais de um, basta
% repetir o comando quantas vezes forem necessárias
%\coorientador{Prof. Dr. Ciclano de Tal}
%\coorientador[fem]{Profª. Drª. Beltrana de Tal}

% A página de rosto da versão para depósito (ou seja, a versão final
% antes da defesa) deve ser diferente da página de rosto da versão
% definitiva (ou seja, a versão final após a incorporação das sugestões
% da banca).
\defesa{
  nivel=mestrado, % mestrado, doutorado ou tcc
  % É a versão para defesa ou a versão definitiva?
  definitiva,
  % É qualificação?
  %quali,
  programa={Computer Science},
  membrobanca={Profª. Drª. Fulana de Tal (orientadora) -- IME-USP [sem ponto final]},
  % Em inglês, não há o "ª"
  %membrobanca{Prof. Dr. Fulana de Tal (advisor) -- IME-USP [sem ponto final]},
  membrobanca={Prof. Dr. Ciclano de Tal -- IME-USP [sem ponto final]},
  membrobanca={Profª. Drª. Convidada de Tal -- IMPA [sem ponto final]},
  % Se não houver, remova
  apoio={During the development of this thesis, the author received financial assistance from CAPES},
  local={São Paulo},
  data=2017-08-10, % YYYY-MM-DD
  % Se quiser estabelecer regras diferentes, converse com seu
  % orientador
  direitos={I allow partial or total reproduction of this work, by any conventional or electronic way, for research and study purposes, as long as the source is cited.}
  % Isto deve ser preparado em conjunto com o bibliotecário
  %fichacatalografica={nome do autor, título, etc.},
}

% As palavras-chave são obrigatórias, em português e
% em inglês. Acrescente quantas forem necessárias.
\keyword{Compilers}
\keyword{GCC}
\keyword{Parallel Computing}
\keyword{Parallel Compilation}
\keyword{LTO}
\keyword{Link Time Optimization}


\palavrachave{Compiladores}
\palavrachave{GCC}
\palavrachave{Computação Paralela}
\palavrachave{Compilação em Paralelo}
\palavrachave{LTO}
\palavrachave{Link Time Optimization}

% O resumo é obrigatório, em português e inglês.
\resumo{
Compiladores são grandes programas destinados a converter código em uma
linguagem $A$ para outra linguagem $B$. Grandes compiladores de software livre
ainda fazem essa conversão de maneira sequencial, tirando pouco proveito do
poder computacional de máquinas manycore, sendo a compilação feita em paralelo
pelo número de arquivos disponíveis no projeto. Isso configura um gargalo de
compilação, principalmente se o projeto utilizar \textit{blobs} grandes que
precisam ser compilados. Nós propomos duas alternativas para melhorar a
performance em máquinas \textit{manycores} no GCC: Paralelização via threads,
ou usando o mecanismo do Link Time Optimization (LTO), para paralelizar a
aplicação das otimizações intraprocedurais do compilador, discutindo suas
vantagens e desvantagens de cada metodologia. Embora usemos o GCC como base,
essas técnicas são gerais o suficiente para serem aplicadas em qualquer
compilador. Conseguimos \textit{speedups} de até $3.53\times$ ao compilar
arquivos individuais, e até $35\%$ de \textit{speedup} ao compilar projetos
inteiros usando um destes métodos, em contraste com utilizando apenas o
\texttt{make} em paralelo. Também apresentamos trabalhos relacionados ao
tópico, indo de \textit{parsing} em paralelo até otimizações em paralelo, além
de um resumo sobre a teoria de \textit{parsing} e compilação.
}

\abstract{
Compilers are enormous software designed to translate code in a language $A$ to
another language $B$. Huge Open Source compilers still do this translation
sequentially, taking little advantage of computational power from manycore
machines, with parallel compilation only regarding the number of files
available in the project. This results in a compilation bottleneck, mainly if
the project uses large blobs that require compilation. We propose two
alternatives to improve the performance in manycore machines in GCC:
parallelize using threads, or using the Link Time Optimization (LTO), to
parallelize the application of intraprocedural optimizations in the compiler,
discussing the advantages and disadvantages of each method. Although we use GCC
as a basis, these techniques are general enough to be applied to any compiler.
We managed a speedup of up to $3.53\times$ when compiling individual files, and
up to $35\%$ of speedup when compiling entire projects contrasting with
\texttt{make} alone. We also present the related works on the topic, from
parallel parsing to parallel optimizations, as well as a summary about parsing
and compiler theory.
}
