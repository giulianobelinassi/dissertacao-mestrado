\chapter{Conclusions}
\label{chap:conclusions}

In this chapter, we will use our results to tackle the Research Questions
presented in Chapter X. 

We start by approaching the \textbf{RQ1}. We show in Section \ref{sec:profile}
that the application of the intraprocedural optimizations and code generation
is a good candidate for parallelization since in GCC it takes 75\% of
compilation time.  For the parser, the related works show that the parser can
also be the target for parallelization, but our experiments show that the time
required for this step is minimal.  The interprocedural optimizations, on the
other hand, are a challenge and if possible may also considerably speed up the
LTO in manycore machines. 

We will now reaproach the Research Questions presented in Chapter
\ref{cap:introducao}.  For \textbf{RQ1}, we show that parallelizing the
intraprocedural optimizations can lead a speedup of up to $4\times$, and we
show a reliable way of doing that on Subsection \ref{sec:parallel_lto}.  As for
\textbf{RQ2}, we presented two ways of parallelizing a compiler, one with
$1.68\times$ speedup and another with $2.4\times$ speedup on sample files.

Our first approach in parallelization involved the application of intraprocedural
optimizations in parallel to each function, using threads. Although this
predicted a $1.68\times$ speedup on GCC, its implementation showed to be
difficult because of the heavy use of global variables in the software.  We
even got to a point where analyzers such as helgrind and thread sanitizer were
not detecting the race conditions due to the race window being too small.
Therefore, this approach may not be adequate for already existing
industrial-scale compilers for adding parallelism.

However, if we are designing a compiler from the ground up, this approach may
be better from a software engineering perspective, as it becomes easier to add
multithreading to a part of the code that was, at least at some point, designed
with this in mind. We also get the advantage of avoiding the overheads
associated with process creation, such as page copying, etc.

Threading the compiler may also be the only option when parallelizing the
parser or interprocedural analysis, otherwise, there would be a huge amount of
interprocess communication. However, Figure \ref{fig:parallel_estimate} shows
that parsing is not a computer-intensive task. This is already parallel in LTO
by reading the multiple files, therefore this may not be of concern on present
days nor in the short future. However, the interprocess analysis is of concern,
once it is the only sequential step on LTO and may dominate the compilation
time on manycore machines just by being sequential.

On our second approach, we show how to use the already-existing LTO engine in
GCC to compile single files in parallel. The clear advantage of this approach
is how fast we got the first results when compared to threading the compiler,
and how fast we got it into a complete working condition (bootstrap the
compiler, testsuite acceptance, fuzzer testing).

Although the results were better than the threaded version ($1.88\times$ vs. $2.4\times$
on a Quad-Core),
one may argue that the reason for that is due to our threaded implementation
not being as optimized as the LTO based version -- which is true -- and the
contrary must be expected because the cost of launching a thread is smaller
than the cost of launching a process plus all the necessary page copying.
Therefore, we may expect even larger speedups when compared to the LTO version
on optimized implementations.

Therefore, as an answer for the \textbf{RQ2}, we show a method of archiving up
to $4\times$ speedup in theory, which when implemented yielded a speedup of up
to $3.52\times$. This method also yielded a speedup of up to $35\%$ when
compiling entire projects (in this case, GCC) when compared to \texttt{make
-j64} alone. The reason for this speedup is the existence of files with large
TU in the GCC project, as shown in Figure \ref{fig:analysis_classical}, and
this result might be interesting when the project uses autogenerated (blob) C++
files. We also found a small speedup on Git, which we find interesting because
it consists of small files. One explanation for this is that it does not have
enough files to fully populate the manycore machine's CPU. Furthermore,
this method shows an clear improvement of compilation time when files
with expected number of instructions larger than $10^3$, which answers
\textbf{RQ4}.


From now on, we will focus on the LTO implementation results to drive our
conclusions. We archived up to $35\%$ speedup when comparing our results with
the \texttt{make -j64} alone, and no significant slowdown when compared to the
sequential version (even if there was, the developer could simply disable the
parallel compilation if this is observed). Furthermore, we show that
partitioning files with TU larger than $10^3$ yields speedups, which answer
\textbf{RQ4}. The reason for this is the existence of files with large TU in
the GCC project, as shown in Figure \ref{fig:analysis_classical}, and this
result might be interesting when the project uses autogenerated (blob) C++
files. We also found a small speedup on Git, which we find interesting because
it consists of small files. One explanation for this is that it does not have
enough files to fully populate the manycore machine's CPU.

As for the electrical energy usage, we show a reduction in power draw by 72\% the
CPU when compiling GCC with our options enabled, which answers \textbf{RQ5}. On
Figure \ref{fig:power}, we observe a significant reduction in the peak power
usage on the parallel version. It may be that there is a point when the
processor is in partial usage (not in full usage) and may draw the same amount
of power as if it is in full usage. Therefore, for better energy efficiency, we
should use the maximum power of the CPU in the shortest amount of time to get a
task done -- and this will also save energy --.

Finally, we expect that our work is relevant for the near future of computing.
As illustrated by \cite{42years}, the CPU manufacturers are expanding on the
number of cores rather than raw sequential speed in recent days. We are already
seeing manufacturers build 64-cores, 128-threads CPUs for workstations, and we
may see CPUs with even more cores in the future unless some new technology
comes up that revolutionizes how CPUs are manufactured and we can see the
sequential performance increase of early 2000 years again.

%%%%%
