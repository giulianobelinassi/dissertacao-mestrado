%%%%%%%%%%%%%%%%%%%%%%% METADADOS (TÍTULO, AUTOR ETC.) %%%%%%%%%%%%%%%%%%%%%%%%%

% Estes comandos definem o título e autoria do trabalho e devem sempre ser
% definidos, pois além de serem utilizados para criar a capa (com o comando
% \maketitle), também são armazenados nos metadados do PDF. O estilo padrão
% de diversos periódicos exige também outros dados, como email, filiação etc.
\title{Título do artigo}
\author{Primeiro Autor\thanks{Filiação do primeiro autor}, Segundo Autor\thanks{Filiação do segundo autor}}

% O pacote hyperref armazena alguns metadados no PDF gerado (em particular,
% o conteúdo de "\title" e "\author"). Também é possível armazenar outros
% dados, como uma lista de palavras-chave.
\hypersetup{
  % Usando "\thanks" no comando \author, hyperref gera uma mensagem
  % de erro. Para contornar isso, repetimos os autores aqui.
  pdfauthor={Primeiro Autor, Segundo Autor},
  pdfkeywords={LaTeX, artigo},
}

% É possível definir como determinadas palavras podem (ou não) ser
% hifenizadas; no entanto, a hifelização automática geralmente funciona bem
\hyphenation{documentclass latexmk Fu-la-no}

% Por padrão, article inclui a data atual; com este comando, você pode
% definir uma data específica, inserir algum outro texto ou, deixando o
% conteúdo em branco, removê-la.
\date{}


%%%%%%%%%%%%%%%%%%%%%%%%%%%%%%%%%%%%%%%%%%%%%%%%%%%%%%%%%%%%%%%%%%%%%%%%%%%%%%%%
%%%%%%%%%%%%%%%%%%%%%%% AQUI COMEÇA O CONTEÚDO DE FATO %%%%%%%%%%%%%%%%%%%%%%%%%
%%%%%%%%%%%%%%%%%%%%%%%%%%%%%%%%%%%%%%%%%%%%%%%%%%%%%%%%%%%%%%%%%%%%%%%%%%%%%%%%

% Gera a "capa" do artigo (geralmente, título, autor etc. sem que haja uma
% quebra de página para o restante do conteúdo)
\maketitle

\begin{abstract}
  Uma variante do arquivo \texttt{tese-exemplo.tex} usando a classe
  \textsf{article}.
\end{abstract}

\section{Introdução}

Se você precisa criar um texto relativamente curto, como um artigo ou
um trabalho de disciplina, este modelo pode servir como base. Observe,
no entanto, que periódicos em geral nas áreas de matemática e computação
costumam ter seus próprios modelos \LaTeX{} (como é o caso da
SBC\footnote{Sociedade Brasileira de Computação}\nocite{sbctemplate});
nesses casos, é melhor utilizá-los e apenas consultar este modelo para
verificar como usar algum recurso específico. Fique atento: alguns modelos
antigos ou de periódicos internacionais podem usar \textsf{latin1} ao
invés de \textsf{utf8} ou mesmo não ter configuração pré-definida para
caracteres acentuados. Além disso, eles muito frequentemente utilizam
bibtex ao invés de biblatex para a geração automática da bibliografia.
